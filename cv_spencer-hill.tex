% Intended LaTeX compiler: xelatex
% Use my "shillcv.cls" file.
\documentclass[letterpaper,11pt]{shillcv}

% Use my shortcuts for commonly used text.
\usepackage{abbrevs}

\title{Spencer Alan Hill}

\begin{document}

\maketitle
\thispagestyle{CVfooter}

\begin{tabular}{@{} L{0.5\textwidth} L{0\textwidth} L{0.5\textwidth} @{}}
Room 733, Marshak Science Building & & \href{mailto:shill1@ccny.cuny.edu}{shill1@ccny.cuny.edu}\\
The City College of New York & & \href{https://shill.ccny.cuny.edu}{https://shill.ccny.cuny.edu}\\
160 Convent Avenue & & office: (212) 650-7027\\
New York, NY 10031 & & \\

\end{tabular}

\vskip1em
\hrule
\vspace{0.5cm}

\section*{EDUCATION}
\begin{longtable}{@{} Y{0.05\textwidth} p{0.05\textwidth} p{0.6\textwidth} p{0.3\textwidth} @{}}
2016 & Ph.D. & Atmospheric and Oceanic Sciences & Princeton University\\
2011 & B.S.  & Atmospheric and Oceanic Sciences; Applied Mathematics & UCLA\\
\end{longtable}
\vskip1em

\section*{PROFESSIONAL APPOINTMENTS}
\begin{longtable}{@{} Y{0.08\textwidth} L{0.92\textwidth} @{}}
2024- & Affiliated Faculty, City University of New York (CUNY) Remote Sensing Earth Systems Institute (CREST)\\
2024- & Doctoral Faculty, \dept/ of Physics, CUNY Graduate Center\\
2023- & Doctoral Faculty, \dept/ of Earth and Environmental Sciences, CUNY Graduate Center\\
2023- & Affiliated Faculty, Program in Earth System Science and Environmental Engineering, City College of New York (CCNY)\\
2023- & Assistant Professor, \dept/ of Earth and Atmospheric Sciences, CCNY\\
2021- & Adjunct Associate Research Scientist, Lamont-Doherty Earth Observatory, Columbia University\\
2021-23 & Associate Research Scholar, Program in Atmospheric and Oceanic Sciences, Princeton University\\
2019-21 & Postdoctoral Research Scientist, Lamont-Doherty Earth Observatory, Columbia University\\
2016-19 & Postdoctoral Research Scientist, dual appointment, Division of Geological and Planetary \mbox{Sciences}, California Institute of Technology, and \dept/ of Atmospheric and Oceanic \mbox{Sciences}, UCLA\\
2011-16 & Graduate Research Assistant, Program in Atmospheric and Oceanic Sciences, Princeton \mbox{University}
\end{longtable}
\vskip1em

\section*{GRANTS AWARDED}
\begin{longtable}{@{} Y{0.05\textwidth} L{0.95\textwidth} @{}}
2025 & ``Machine learning applied to open problems in Earth and Atmospheric Sciences.''  Principal Investigator. 1 yr, \$10,000.  CUNY Research in the Classroom program.\\
     & Principal Investigator.  1 yr, \$8,000.  CUNY Critical A.I. Literacy Institute (CALI).\\
     & ``A first-ever quantification of moist static energy transports by tropical cyclones.''  Principal Investigator.  1 yr, \$5,999.53.  Professional Staff Congress (PSC) of CUNY Research Award, Cycle 56.\\
2024 & ``The Climate Lighthouse: DOE-CCNY Urban Climate Hazard Resilience Center.''  Co-Investigator.  3 yr, \$998,973.  U.S. \dept/ of Energy, Office of Biological \& Environmental Research, Climate Resilience Centers program.\\
2023 & ``How do energy fluxes link precipitation variability across the tropical weather-climate continuum?''  Principal Investigator.  2 yr, \$399,465.  NSF Climate and Large-Scale Dynamics, Award AGS-2411723.\\
     & \hspace{1cm}{\color{gray}[Transfer to CCNY from Princeton of original AGS-2123327 awarded 2021 (3 yr, \$682,038)]}\\

\end{longtable}

\section*{EXTRAMURAL FUNDING PROPOSALS CURRENTLY UNDER REVIEW}
\begin{longtable}{@{} Y{0.05\textwidth} L{0.95\textwidth} @{}}
2025 & ``Improving Understanding of Wildfire Smoke Hazards for the Northeast U.S.''  Principal Investigator.  2 yr, \$206,455.  NASA Mentorship and Opportunities in STEM with Academic Institutions for Community Success (MOSAICS) Seed Funding.\\
2024 & ``Moist static energy transports by tropical cyclones.''  Principal Investigator.  2 yr, \$30,000.  Lamont-Doherty Earth Observatory INSPIRE Fellows program.\\
     & ``Collaborative Research: Extreme Rainfall in South Asia: Mechanisms and Variability.''  Principal Investigator.  3 yr, \$408,862.  NSF Climate and Large-scale Dynamics.\\
\end{longtable}
\vskip1em

\section*{POSTDOCTORAL AND GRADUATE FELLOWSHIPS AWARDED}
\begin{longtable}{@{} Y{0.05\textwidth} L{0.95\textwidth} @{}}
2019 & Columbia University Earth Institute Postdoctoral Research Fellowship (2019-21)\\
2016 & California Institute of Technology Foster and Coco Stanback Postdoctoral Research Fellowship (deferred to 2018-19)\\
     & NSF Atmospheric and Geospace Sciences Postdoctoral Research Fellowship (2016-18)\\
2013 & \dept/ of Defense National Defense Science and Engineering Graduate Research Fellowship (2013-16)\\
\end{longtable}
\vskip1em

\section*{PUBLICATIONS}

\subsection*{Articles submitted/under review/in revision}
\begin{longtable}{@{} L{\textwidth} @{}}
\textbf{Hill, SA}, D Zamir-Meyers, AH Sobel, M Biasutti, MA Cane, MK Tippett, F Ahmed.  ``More extreme Indian monsoon daily rainfall in El Ni\~no summers.''  In revision, \emph{Science}.  \href{https://arxiv.org/abs/2404.12419}{arxiv:2404.12419}.\\
\end{longtable}

\subsection*{Refereed journal articles}
\begin{longtable}{@{} Y{0.05\textwidth} >{\color{black}} p{0.04\textwidth} @{} L{0.91\textwidth} @{}}
2025 & 26. & \textbf{Hill, SA}, S Bordoni, JL Mitchell, JM Lora.  ``Interpreting seasonal and interannual Hadley cell descending edge migrations via the cell-mean Rossby number.''  Accepted, \emph{J. Climate}.  \href{https://arxiv.org/abs/2411.14544}{arXiv:2411.14544}\\
2024 & 25. & Mossel, C, \textbf{S Hill}, N Samal, J Booth, N Devineni. ``Increasing Extreme Hourly Precipitation Risk for New York City after Hurricane Ida.''  \emph{Scientific Reports}, \textbf{14}, 27947.  doi:\href{https://doi.org/10.1038/s41598-024-78704-9}{10.1038/s41598-024-78704-9}.\\
     & 24. & Byrne MP, G Hegerl, J. Scheff, \ldots, \textbf{SA Hill}, et al. ``Theory and the future of land-climate science.''  \emph{Nature Geoscience}, \textbf{17}, 1079-1086.  doi:\href{https://doi.org/10.1038/s41561-024-01553-8}{10.1038/s41561-024-01553-8}.\\
     & 23. & Clark, JP, P Lin, \textbf{SA Hill}. ``ITCZ Response to Disabling Parameterized Convection in Global Fixed-SST Aquaplanet Simulations at 50 km and 6 km Resolutions.'' \emph{J.  Adv. in Mod. Earth Sys.}, \textbf{16}, e2023MS003968.  doi:\href{https://doi.org/10.1029/
2023MS003968}{10.1029/2023MS003968}.\\
2023 & 22. & Zurita-Gotor P, IM Held, TM Merlis, CY Chang, \textbf{SA Hill}, CG MacDonald.  ``Non-uniqueness in ITCZ latitude due to radiation-circulation coupling in an idealized GCM.'' \emph{J.  Adv. in Mod. Earth Sys.}, \textbf{15}(10), e2023MS003736.  doi: \href{https://doi.org/10.1029/2023MS003736}{10.1029/2023MS003736}.\\
     & 21. & Biasutti M, M Ting, \textbf{SA Hill}. ``The dynamics and changes of the world’s monsoons.''  \emph{Physics Today}, \textbf{76}(9), 32-38.  doi: \href{https://doi.org/10.1063/PT.3.5308}{10.1063/PT.3.5308}.\\
     & 20. & Moscoso JE, RE Tripoli, S Chen, H Gonzalez, \textbf{SA Hill}, N Khoo, TL Lonner, JM Aurnou. ``Low-cost table-top experiments for teaching
multi-scale geophysical fluid dynamics.''  \emph{Frontiers in Marine Science}, \textbf{10}, 1192056.  doi:  \href{https://doi.org/10.3389/fmars.2023.1192056}{10.3389/fmars.2023.1192056}.\\
     & 19. & Ahmed F, JD Neelin, \textbf{SA Hill}, K Schiro, and H Su. ``A process model for ITCZ narrowing under warming highlights clear-sky water vapor feedbacks and gross moist stability changes in AMIP models.'' \emph{J. Climate}, \textbf{36}, 4913-4931.  doi: \href{https://doi.org/10.1175/JCLI-D-22-0689.1}{10.1175/JCLI-D-22-0689.1}.\\
     & 18. & Mahfouz N, \textbf{SA Hill}, H Guo, and Y Ming. ``The radiative and cloud responses to sea salt aerosol geoengineering in GFDL models.''  \emph{Geophys.\@ Res.\@ Lett.}, \textbf{50}, e2022GL102340.  doi: \href{https://doi.org/10.1029/2022GL102340}{10.1029/2022GL102340}.\\
2022 & 17. & \textbf{Hill SA}, NJ Burls, A Fedorov, and TM Merlis.  ``Symmetric and antisymmetric components of polar-amplified warming.''  \emph{J. Climate}, \textbf{35}, 3157-3172.  doi: \href{https://doi.org/10.1175/JCLI-D-20-0972.1}{10.1175/JCLI-D-20-0972.1}.\\
     & 16. & \textbf{Hill SA}, S Bordoni, and JL Mitchell.  ``A theory for the Hadley cell descending and ascending edges throughout the annual cycle.''  \emph{J. Atmos. Sci.}, \textbf{79}, 2515-2528.  doi: \href{https://doi.org/10.1175/JAS-D-21-0328.1}{10.1175/JAS-D-21-0328.1}.\\
     & 15. & Biasutti M, \textbf{SA Hill}, and A Voigt.  ``The Effect Of An Equatorial Continent On The Tropical Rain Belt. Part 2: Summer Monsoons.''  \emph{J. Climate}, \textbf{35}, 3091–3107.  doi: \href{https://doi.org/10.1175/JCLI-D-21-0588.1}{10.1175/JCLI-D-21-0588.1}\\
     & 14. & \textbf{Hill SA}, AH Sobel, M Biasutti, and MA Cane.  ``On the all-India rainfall index and sub-India rainfall heterogeneity.''  \emph{Geophys. Res. Lett.}, \textbf{49}, e2021GL096541, 10 pp.  doi: \href{https://doi.org/10.1029/2021GL096541}{10.1029/2021GL096541}.\\
2021 & 13. & \textbf{Hill SA}, S Bordoni, and JL Mitchell.  ``Solsticial Hadley Cell ascending edge theory from supercriticality.''  \emph{J. Atmos. Sci}, \textbf{78}, 1999-2011.  doi: \href{https://doi.org/10.1175/JAS-D-20-0341.1}{10.1175/JAS-D-20-0341.1}.\\
     & 12. & Mitchell, JL and \textbf{SA Hill}.  ``Constraints from invariant
subtropical vertical velocities on the scalings of Hadley cell strength and
downdraft width with rotation rate.''  \emph{J. Atmos. Sci}, \textbf{78}, 1445-1463.  doi: \href{https://doi.org/10.1175/JAS-D-20-0191.1}{10.1175/JAS-D-20-0191.1}\\
2020 & 11. & \textbf{Hill SA}, S Bordoni, and JL Mitchell.
``Axisymmetric Hadley Cell theory with a fixed tropopause temperature rather
than height.'' \emph{J. Atmos. Sci.}, \textbf{77}, 1279-1294.  doi: \href{https://doi.org/10.1175/JAS-D-19-0169.1}{10.1175/JAS-D-19-0169.1}.\\
2019 & 10. & \textbf{Hill SA}.  ``Theories for past and future monsoon rainfall
changes.'' \emph{Curr. Clim. Change Rep.}, \textbf{5}, 160-171.  doi: \href{https://doi.org/10.1007\%2Fs40641-019-00137-8}{10.1007\%2Fs40641-019-00137-8}.\\
     & 9. & \textbf{Hill SA}, S Bordoni, and JL Mitchell.
``Axisymmetric constraints on cross-equatorial Hadley cell extent.''
\emph{J. Atmos. Sci.}, \textbf{76}, 1547-1564.  doi: \href{https://doi.org/10.1175/JAS-D-18-0306.1}{10.1175/JAS-D-18-0306.1}.\\
2018 & 8. & \textbf{Hill SA}, JM Lora, N Khoo, SP Faulk, and
JM Aurnou.  ``Affordable rotating fluid demonstrations for
geoscience education: The \emph{DIYnamics} project.''  \emph{Bull.
Am. Met. Soc.}, \textbf{99}, 2529-2538.  doi: \href{https://doi.org/10.1175/BAMS-D-17-0215.1}{10.1175/BAMS-D-17-0215.1}.\\
     & 7. & \textbf{Hill SA}, Y Ming, and M Zhao.  ``Robust responses of the
Sahelian hydrological cycle to global warming.''  \emph{J. Climate}, \textbf{31}, 9793-9814.  doi: \href{https://doi.org/10.1175/JCLI-D-18-0238.1}{10.1175/JCLI-D-18-0238.1}.\\
     & 6. & Smyth J, \textbf{SA Hill}, and Y Ming.  ``Simulated responses of
the West African monsoon and zonal-mean tropical precipitation to early
Holocene orbital forcing.''  \emph{Geophys. Res. Lett.}, \textbf{45},
12,049-12,057.  doi: \href{https://doi.org/10.1029/2018GL080494}{10.1029/2018GL080494}.\\
2017 & 5. & \textbf{Hill SA}, Y Ming, IM Held, and M Zhao.  ``A moist
static energy budget-based analysis of the Sahel rainfall response to uniform
oceanic warming.''  \emph{J. Climate}, \textbf{30}, 5637-5660.  doi: \href{https://doi.org/10.1175/JCLI-D-16-0785.1}{10.1175/JCLI-D-16-0785.1}.\\
     & 4. & Brown PT, Y Ming, W Li, and \textbf{SA Hill}.  ``Change
in the magnitude and mechanisms of unforced low-frequency surface temperature
variability in a warmer climate.''  \emph{Nature Climate Change}, \textbf{7}, 743-748.  \href{https://doi.org/10.1038/nclimate3381}{10.1038/nclimate3381}.\\
     & 3. & Jeevanjee N, P Hassanzadeh, \textbf{SA Hill}, and A Sheshadri.  "A perspective on climate model hierarchies."  \emph{J.  Adv. in Mod. Earth Sys.}, \textbf{9}, 1760-1771.  doi: \href{https://doi.org/10.1002/2017MS001038}{10.1002/2017MS001038}.\\
2015 & 2. & \textbf{Hill SA}, Y Ming, and IM Held.  ``Mechanisms of forced
tropical meridional energy flux change.''  \emph{J. Climate}, \textbf{28}, 1725-1742.  doi: \href{http://dx.doi.org/10.1175/JCLI-D-14-00165.1}{10.1175/JCLI-D-14-00165.1}.\\
& & \hspace{1cm} Corrigendum: \href{https://dx.doi.org/10.1175/JCLI-D-16-0485.1}{https://dx.doi.org/10.1175/JCLI-D-16-0485.1}.\\
2012 & 1. & \textbf{Hill SA} and Y Ming.  ``Nonlinear climate response to regional
brightening of tropical marine stratocumulus.''  \emph{Geophys. Res. Lett.},
\textbf{39}, L15707, 5 pp.  doi: \href{http://dx.doi.org/10.1029/2012GL052064}{10.1029/2012GL052064}.\\
\end{longtable}

\subsection*{Book reviews}
\begin{longtable}{@{}  Y{0.05\textwidth} L{0.95\textwidth} @{}}
2012 & \textbf{Hill SA}  ``A head in the clouds elucidates climate'' (book
review of \emph{Atmosphere, Clouds, and Climate} by David Randall).  \emph{Science}, \textbf{337},
1 pp., doi: \href{http://dx.doi.org/10.1126/science.1225615}{10.1126/science.1225615}.\\
\end{longtable}

\subsection*{Other publications}
\begin{longtable}{@{}  Y{0.08\textwidth} L{0.92\textwidth} @{}}
2022 & Biasutti M, M Ting, and \textbf{SA Hill}.  ``How the south Asian monsoon is changing in a warming climate.''  Guest post, \emph{Carbon Brief}.  \sep/ 15th. \href{https://www.carbonbrief.org/guest-post-how-the-south-asian-monsoon-is-changing-in-a-warming-climate/}{https://www.carbonbrief.org/guest-post-how-the-south-asian-monsoon-is-changing-in-a-warming-climate/}.\\

2017- &  \raggedright Fourteen blog posts for the DIYnamics blog.  Available at \href{https://diynamics.github.io/blog/author/spencer-hill.html}{https://diynamics.github.io/blog/author/spencer-hill.html}.\\

% SAH note: this whole blog appears to have gone offline.
% So it's not appropriate to include anymore.
% 2017 & \textbf{Hill, Spencer A.} and Spencer K. Clark.  ``What’s needed for the Future
% of AOS Python?  Tools for Automating AOS Data Analysis and Management.''
% Invited guest blog post on ``PyAOS'' blog.
% \href{http://pyaos.johnny-lin.com/?p=1546}{http://pyaos.johnny-lin.com/?p=1546}.\\

\end{longtable}

\section*{AWARDS AND HONORS}
\begin{longtable}{@{}  Y{0.05\textwidth} L{0.95\textwidth} @{}}
2021 & Columbia University Nominee, Blavatnik Regional Award for Young Scientists\\
2012 & Princeton University Elliotte Robinson Little '25 Fellowship\\
2011 & American Meteorological Society Annual Meeting Climate Change Travel Scholarship\\
     & NSF Graduate Research Fellowship Honorable Mention\\
     & UCLA Magna Cum Laude and College Honors graduation distinctions\\
2009 & National Oceanic and Atmospheric Administration Ernest F. Hollings Undergraduate Scholarship\\
2007 & United States Presidential Scholar, conferred by the U.S. \dept/ of Education Commission on Presidential Scholars
\end{longtable}
\vskip1em

\section*{INVITED COLLOQUIA AND SEMINARS}
\begin{longtable}{@{} Y{0.05\textwidth} L{0.95\textwidth} @{}}
2025 & 3 lectures, International Center for Theoretical Physics Workshop on Global Monsoons: Theory, Models and Observations, Universidad Nacional Aut\'{o}noma de M\'{e}xico, Mexico City\\
     & \hspace{1cm} ``Introduction to tropical general circulation and the global monsoons.'' \jan/ 20th.\\
     & \hspace{1cm} ``Theories for past and future monsoon rainfall changes.'' \jan/ 21st.\\
     & \hspace{1cm} ``Interannual variability in extreme daily monsoon rainfall.'' \jan/ 22nd.\\  
2024 & \dept/ of  Atmospheric and Oceanic Sciences, University of Wisconsin, Madison, \octob/ 21st.\\
     & \dept/ of Earth and Environmental Sciences, City University of New York Graduate Center, \apr/ 4th.\\
     & Division of Ocean and Climate Physics seminar series, Lamont-Doherty Earth Observatory, \mar/ 29th.\\
     & \dept/ of Mechanical Engineering, City College of New York, \feb/ 22nd.\\
     & \dept/ of Physics, City College of New York, \feb/ 7th.\\
2023 & \dept/ of Earth and Planetary Science, Yale University, \mar/ 8th.\\
     & Atmosphere, Ocean, and Climate Dynamics seminar series, Yale University, \mar/ 9th.\\
     & \dept/ of Atmospheric and Oceanic Sciences, UCLA, \mar/ 1st.\\
     & \dept/ of Earth and Atmospheric Sciences, City College of New York, \feb/ 14th.\\
2022 & Equilibrium Climate Sensitivity \& Cloud Feedback Virtual Symposium, \mar/ 24th.\\
2021 & Atmosphere, Ocean, and Climate Dynamics seminar series, Yale University, \feb/ 25th.\\
     & Meteorology Seminar Series, \dept/ of Earth, Atmospheric, and Oceanic Sciences, Florida State University, \feb/ 18th.\\
2019 & Monthly Climate Meeting, Earth Research Institute, University of California -- Santa Barbara, \jan/ 7th.\\
2018 & Division of Ocean and Climate Physics, Lamont-Doherty Earth Observatory, \feb/ 16th.\\
     & NOAA Geophysical Fluid Dynamics Laboratory, \feb/ 14th.\\
2016 & Center for Atmosphere Ocean Science, New York University, \nov/ 9th.\\
     & Natural and Behavioral Sciences Lecture, Gaede Institute for the Liberal Arts, Westmont College, \octob/ 13th.\\
2015 & Student seminar series, Center for Atmospheric and Oceanic Sciences, New York University, \mar/ 6th.\\
2014 & \dept/ of Geophysics, Yale University, \octob/ 9th.\\
\end{longtable}
\vskip1em


\section*{CONFERENCE ACTIVITIES}

\subsection*{Chaired sessions}
\begin{longtable}{@{} Y{0.05\textwidth} L{0.95\textwidth} @{}}
2024 & ``Tropical Atmospheric Dynamics.'' AMS Atmospheric and Oceanic Fluid Dynamics Conference, Burlington, VT, \jun/ 28th.\\
2023 & ``Large-Scale Atmospheric Dynamics and Climate: Jet Streams, Storm Tracks, Stationary Waves, and Monsoons IV'' AMS Annual Meeting, Denver, CO, \jan/ 10th.\\
2022 & ``Monsoon Dynamics: Variability, Change, and Impacts I and II'' AMS Annual Meeting (virtual), \jan/ 26th-27th.\\
2018 & ``Monsoons: Observations, Subseasonal, Seasonal, and Interannual to Decadal Variability, Forecast, Climate Change, and Extremes III.''  AGU Fall Meeting, Washington, D.C., \dec/ 11th. \\
2017 & ``Idealized approaches to the atmospheric and oceanic circulation II.'' American Meteorological Society 21st Conference on Atmospheric and Oceanic Fluid Dynamics, Portland, OR, \jun/ 26th.\\
2016 & ``Tropical circulations and their sensitivities to changes in climate I.''  AGU Fall Meeting, San Francisco, CA, \dec/ 16th.\\
     & ``Tropical convection and radiative convective equilibrium.''  World Climate Research Programme Model Hierarchies Workshop, Princeton, NJ, \nov/ 3rd.\\
\end{longtable}

\subsection*{Invited Conference Talks}
\begin{longtable}{@{} Y{0.05\textwidth} L{0.95\textwidth} @{}}
2023 & ``More extreme Indian monsoon daily rainfall in El Ni\~no summers.'' AMS Annual Meeting, Denver, CO, \jan/ 9th.\\
2022 & ``Indian summer monsoon rainfall interannual variability: seasonal-mean and daily extremes.'' Continental Climate Change Workshop, St. Andrews, Scotland, \jun/ 8th.\\
2019 & ``Toward an analytical, predictive theory for the location of Hadley and monsoonal cell ascending branches.''  AGU Fall Meeting, San Francisco, CA, \dec/ 10th.\\
2016 & ``\texttt{infinite-diff} and \texttt{animal-spharm}: \texttt{xarray}-based finite differencing and spherical harmonics.''  Columbia University Python and Atmospheric and Oceanic Sciences Workshop, New York, NY, \nov/ 12th.\\
2015 & ``Towards constraining Sahel rainfall responses to global mean temperature changes.''   Linde Center for Global Environmental Science, California Institute of Technology, Monsoons: Past, Present and Future workshop, Pasadena, CA, \may/ 21st.\\
\end{longtable}

\subsection*{Other Conference Presentations}
\begin{longtable}{@{} Y{0.05\textwidth} L{0.95\textwidth} @{}}
2025 & ``More extreme Indian monsoon daily rainfall in El Ni\~no summers.''  Talk.  Eighth World Meteorological Organization Workshop on Monsoons.  Virtual.  \mar/ 18th.\\
2024 & ``Explaining Hadley cell extent across timescales via the cell-mean Rossby number.'' Talk.  AGU Annual Meeting.  Washington, DC.  \dec/ 13th.\\
     & ``More extreme Indian monsoon daily rainfall in El Ni\~no summers.''  Talk.  4th Workshop on Convective Organization.  International Centre for Theoretical Physics, Trieste, Italy.  \jul/ 11th.\\
     & ``Investigating Hadley cell extent across timescales via the cell-mean Rossby number.''  Talk.  AMS Atmospheric and Oceanic Fluid Dynamics meeting.  Burlington, VT.  \jun/ 28th.\\
     & ``More extreme Indian monsoon daily rainfall in El Niño summers.''  Poster.  CFMIP Meeting.  Boston, MA.  \jun/ 4th.\\
     & ``The Hadley Cells Across Seasons and the Solar System''  Talk.  AMS Annual Meeting.  Baltimore, MD. \jun/ 17th.\\
2023 & ``More extreme Indian monsoon daily rainfall during El Niño summers.''  Poster.  AGU Fall Meeting.  San Francisco, CA. \dec/ 12th\\
2022 & ``Hadley cell extent across seasons and the solar system.''  Talk.  AMS Atmospheric and Oceanic Fluid Dynamics Meeting.  Breckenridge, CO. \jun/ 17th.\\
& ``Symmetric and antisymmetric components of polar-amplified warming.'' Poster.  U.S. CLIVAR Pattern Effect Workshop, Boulder, CO.  \may/ 11th.\\
& ``On the All-India Rainfall Index, Sub-India Rainfall Heterogeneity, ENSO, and Teleconnections from the Indian Ocean''  Talk.  AMS Annual Meeting (virtual).  \jan/ 26th.\\
& ``Symmetric and Antisymmetric Components of Polar-Amplified Warming.''  Talk.  AMS Annual Meeting (virtual).  \jan/ 26th.\\
2021 & ``New scalings for the ascending and descending branch positions of the solsticial Hadley cells in planetary atmospheres including Titan''  Poster.  AGU Fall Meeting.  New Orleans, LA.  \dec/ 16th.\\
 & ``On the all-India rainfall index, sub-India rainfall heterogeneity, ENSO, and teleconnections from the Indian Ocean''  Talk.  AGU Fall Meeting.  New Orleans, LA.  \dec/ 15th.\\
& ``Connecting sub-India, sub-seasonal monsoon rainfall variability with all-India, all-summer monsoon rainfall.'' AMS Annual Meeting (virtual).  Talk.  \jan/ 14th.\\
2020 & ``Sub-India summer monsoon rainfall variability and its implications for all-India summer monsoon rainfall prediction.''  AGU Fall Meeting (virtual).  Poster.  \dec/ 1st-17th.\\
2019 & ``Simulated polar amplification and its causes on decadal to millennial timescales.''  Poster.  AGU Fall Meeting.  San Francisco, CA.  \dec/ 10th.\\
     & ``Modernizing Axisymmetric Hadley Cell and Monsoon Theory.''  Talk.  AMS 22nd Conference on Atmospheric and Oceanic Fluid Dynamics.  Portland, ME.  \jun/ 25th.\\
2018 & ``What sets the locations of the solsticial cross-equatorial Hadley cell edges?''  Talk.  AGU Fall Meeting.  Washington, DC.  \dec/ 13th.\\
     & ``Towards transient simulation of the Green Sahara onset and demise through idealized modeling of vegetation-land-atmosphere interactions.''  Poster.  17th Swiss Climate Summer School: Earth system variability through time.  Grindelwald, Switzerland.  \aug/ 28th.\\
     & ``What Determines the ITCZ Position During Solsticial Seasons on Earth and Other Planets?''  Talk.  AMS 33rd Conference on Hurricanes and Tropical Meteorology.  Ponte Vedra, FL.  \apr/ 16th.\\
2017 & ``Dry Rainbelts: Understanding Boundary Layer Controls on the ITCZ Using a Dry Dynamical Core.''  Talk.  AGU Fall Meeting.  New Orleans, LA.  \dec/ 14th.\\
     & ``Towards transient simulation of the Green Sahara onset and demise through idealized modeling of vegetation-land-atmosphere interactions.''  Poster.  Gordon Research Conference on Radiation and Climate.  Bates College, Lewiston, ME.  \jul/ 19th.\\
     & ``Control of convergence zone migrations by planetary parameters.''  Poster.  AMS 21st Conference on Atmospheric and Oceanic Fluid Dynamics.  Portland, OR.  \jun/ 27th.\\
     & ``Automate your climate and weather data analysis with aospy.''  Talk.  AMS Annual Meeting, Seattle, WA.  \jan/ 24th.\\
     & ``Energetic and precipitation responses in the Sahel to sea surface temperature perturbations.''  Talk.  AMS Annual Meeting, Seattle, WA.  \jan/ 24th.\\
2016 & ``Robust drying influence of mean ocean surface warming on The Sahel and implications for constraining future rainfall change.''  Poster.  AGU Fall Meeting, San Francisco, CA.  \dec/ 16th.\\

     & ``A hierarchy of perturbation complexites: Case study of Sahel rainfall response to global warming''  Poster.  WCRP Model Hierarchies Workshop, Princeton University, Princeton, NJ.  \nov/ 2nd.\\
2015 & ``Towards constraining future rainfall in the Sahel using the moist static energy budget.'' Talk.  AGU Fall Meeting, San Francisco, CA.  \dec/ 14th.\\
     & ``Convection scheme, cloud, and stability effects on Sahel rainfall response to uniform warming.''  Poster.  AMS Annual Meeting, Phoenix, AZ. \jan/ 6th.\\
2014 & ``Convection scheme, cloud, and stability effects on Sahel rainfall response to uniform warming.''  Poster.  AGU Fall Meeting, San Francisco, CA.  \dec/ 15th.\\
     & ``Mechanisms of forced tropical meridional energy flux change.''  Poster.  Latsis Symposium, ETH Zurich, Zurich, Switzerland. \jun/ 19th.\\
     & ``Mean and extreme tropical precipitation changes caused by the uniform and spatially varying components of anthropogenic forcing.''  Talk.  AMS Annual Meeting, Atlanta, GA.  \feb/ 5th.\\
2013 & ``Mechanisms of forced tropical meridional energy flux change.''  Talk.  AGU Fall Meeting, San Francisco, CA. \dec/ 13th.\\
     & ``Mechanisms of forced tropical meridional energy flux change.''  Talk.  Graduate Climate Conference, Woods Hole Oceanographic Institution, Woods Hole, MA.  \nov/ 2nd.\\
     & ``Mechanisms of forced tropical meridional energy flux change.'' Poster presentation.  Gordon Research Conference, Colby-Sawyer College, New London, NH.  \jul/ 9th.\\
2012 & ``Climate response to a geoengineered brightening of subtropical marine boundary clouds.''  Poster.  11th Annual Student Conference at the AMS Annual Meeting, New Orleans, LA.  \jan/ 22nd.\\
2010 & ``Climate response to a geoengineered brightening of subtropical marine boundary clouds.''  Poster.  San Francisco, CA.  \dec/ 14th.\\
     & ``Climate response to a geoengineered brightening of subtropical marine boundary clouds.'' Talk.  Special Symposium on Aerosols in Geoengineering at the American Association for Aerosol Research 29th Annual Conference.  Portland, OR.  \octob/ 26th.\\
     & ``Investigating climate response to geoengineering using a global climate model.''  Talk.  National Oceanic and Atmospheric Administration Office of Education Science Symposium, Silver Spring, MD.  \aug/ 3rd.\\
\end{longtable}
\vskip1em


\section*{CAMPUS AND DEPARTMENTAL TALKS}
\begin{longtable}{@{} Y{0.05\textwidth} L{0.95\textwidth} @{}}
2023 & Princeton AOS Dynamics Group Meeting.  \feb/ 23rd.\\
2022 & Princeton AOS Dynamics Group Meeting.  \aug/ 11th.\\
     & Princeton AOS Discussion group on Isaac Held's Blog.  \aug/ 1st.\\
     & Princeton AOS Tutorial Series for summer interns.  \jun/ 21st.\\
2019 & Lamont-Doherty Earth Observatory Postdoc Symposium (poster).  \sep/ 11th.\\
2018 & \dept/ of Atmospheric and Oceanic Sciences, UCLA.  \nov/ 7th.\\
2016 & Division of Geological and Planetary Sciences, California Institute of Technology.  \octob/ 26th.\\
     & \dept/ of Atmospheric and Oceanic Sciences, UCLA.  \octob/ 5th.\\
2015 & Dynamics Seminar Series, Program in Atmospheric and Oceanic Sciences, Princeton University.  \mar/ 13th.\\
2012 & Student/Postdoc Seminar Series, Princeton AOS.  \feb/ 28th.\\
2011 & Graduate Research Symposium, \dept/ of Geosciences, Princeton University. \nov/ 11th.\\
\end{longtable}
\vskip1em


\section*{ADVISING}
\subsection*{Postdocs}
\begin{longtable}{@{}  Y{0.07\textwidth} L{0.93\textwidth} @{}}
2025- & Shreya Keshri, \dept/ of Earth and Atmospheric Sciences, CCNY\\
2024- & Haochang Luo, \dept/ of Earth and Atmospheric Sciences, CCNY\\
\end{longtable}

\subsection*{PhD. students}
\begin{longtable}{@{}  Y{0.07\textwidth} L{0.93\textwidth} @{}}
2025-& Gregory Randazzo, \dept/ of Earth and Environmental Sciences, CUNY Graduate Center\\
2022- & Alexander Parsells, \dept/ of Earth and Environmental Sciences, Columbia University (co-advised with Michela Biasutti)\\
\end{longtable}

\subsection*{Masters students}
\begin{longtable}{@{}  Y{0.07\textwidth} L{0.93\textwidth} @{}}
2024-& Michelle Wagner, \dept/ of Earth and Atmospheric Sciences, CCNY\\
2023-& Gregory Randazzo, \dept/ of Earth and Atmospheric Sciences, CCNY (co-advised with James Booth)\\
\end{longtable}

\subsection*{Undergraduate students}
\begin{longtable}{@{}  Y{0.07\textwidth} L{0.93\textwidth} @{}}
2024-& Edda Hobuss, CCNY\\
2021 & Destiny Zamir Meyers, Columbia University\\
     & Matthew Donahue, Columbia University\\
2020 & Valentina Rojas-Posada, Barnard University (partial funding awarded by competition from Columbia University Earth Institute)\\
2017 & Norris Khoo, UCLA\\
     & Micah Kim, UCLA\\
2016 & Juliet Olsen, UCLA\\
\end{longtable}

\subsection*{Undergraduate summer interns, as co-advisor}
\begin{longtable}{@{}  Y{0.05\textwidth} L{0.95\textwidth} @{}}
2024 & Rachel Ioffe (CCNY IT-ROCS REU program, co-advisor Greg Randazzo)\\
     & Ilvis Cid (CCNY IT-ROCS REU program, co-advisor Greg Randazzo)\\         
2023 & Kittson Hamill (NOAA GFDL, co-advisor Nadir Jeevanjee)\\
2022 & Rea Restugi (NOAA GFDL, co-advisors Akshaya Nikumbh, Yi Ming)\\
2015 & Jane Smyth (NOAA GFDL, co-advisor Yi Ming)\\
2014 & Marjahn Finlayson (NOAA GFDL, co-advisor Yi Ming)\\
2013 & Colin Raymond (NOAA GFDL, co-advisor Yi Ming)\\
\end{longtable}
\vskip1em

\section*{TEACHING ACTIVITIES}
\subsection*{Courses}
\begin{longtable}{@{} Y{0.12\textwidth} L{0.9\textwidth} @{}}
2025 Spring & CCNY EAS 48800/B4800 ``Climate and Climate Change''\\
2024 Fall & CCNY EAS 30000/B1300 ``Earth and Environmental Science Seminar''\\
2024 Spring & CCNY EAS 48800/B4800 ``Climate and Climate Change''\\
2023 Fall & CCNY EAS 42000/A4200 ``Statistical Methods / Quantitative Data Analysis in Earth and Atmospheric Sciences.''\\
\end{longtable}

\subsection*{Teacher training workshops organized}
\begin{longtable}{@{} Y{0.05\textwidth} L{0.95\textwidth} @{}}
2022 & ``Teaching atmosphere, ocean, and planetary fluid dynamic fundamentals vividly with rotating tanks.''  Earth Educators Rendezvous, Twin Cities, MN, \jul/ 14-15th.\\
\end{longtable}

\subsection*{Teaching Assistant}
\begin{longtable}{@{} Y{0.05\textwidth} L{0.95\textwidth} @{}}
2014 & ``Physics of Earth: The Habitable Planet.''  Upper-division undergraduate course.  \dept/ of Geosciences, Princeton University.\\
\end{longtable}

\subsection*{Certifications}
\begin{longtable}{@{} Y{0.05\textwidth} L{0.95\textwidth} @{}}
2016 & Teaching Transcript Certification, McGraw Center for Teaching and Learning, Princeton University.\\
\end{longtable}
\vskip1em

\subsection*{Guest Lectures/Labs}
\begin{longtable}{@{} Y{0.05\textwidth} L{0.95\textwidth} @{}}
2024 & Lecture for ``Selected Topics in Advanced Physics: Fluid Dynamics Applied to Classical \& Quantum Systems'' (PHYS 85200) graduate course, \dept/ of Physics, CUNY Graduate Center.  \nov/ 5th.\\
     & Two lectures for ``ESS Modeling/Databases'' (EAS 30800) undergraduate course, \dept/ of Earth and Atmospheric Sciences, City College of New York. \sep/ 3rd and 5th.\\
2023 & Lecture for ``Climate Thermodynamics and Energy Transfer'' (EESC 4040) graduate course, \dept/ of Earth \& Environmental Science, Columbia University.  \apr/ 25th.\\
2022 & Lab for EAS 30900/B3090 ``Fundamentals of Atmospheric Science'' graduate course, \dept/ of Earth and Atmospheric Sciences, City College of New York.  \dec/ 5th.\\
     & Lab for ``Earth's Environmental Systems: The Climate System'' undergraduate course, \dept/ of Earth and Environmental Sciences, Columbia University.  \feb/ 24th.\\
2018 & Three lectures for ``Introduction to Atmospheric and Oceanic Fluids,'' graduate course, \dept/ of Atmospheric and Oceanic Sciences, UCLA.\\
& \hspace{1cm}``The general circulation of the atmosphere: Energetics.''  \nov/ 13th.\\
& \hspace{1cm}``The general circulation of the atmosphere: Momentum.''  \nov/ 15th.\\
& \hspace{1cm}``The general circulation of the atmosphere: Lab.''  \nov/ 29th.\\
2014 & Two lectures for ``Physics of Earth: The Habitable Planet,'' upper-division undergraduate course, \dept/ of Geosciences, Princeton University.\\
& \hspace{1cm}``Weather: From orbital order, atmospheric chaos.''  \octob/ 14th.\\
& \hspace{1cm}``Climate modeling and climate change projections.''  \dec/ 10th.\\

\end{longtable}
\vskip1em


\section*{SCIENCE OUTREACH AND MEDIA}
\begin{longtable}{@{} Y{0.08\textwidth} L{\textwidth} @{}}
2023 & ``Climate, Chaos and CCNY: A new professor in EAS — Dr. Spencer Hill.''  Interview, \emph{The RICC} (Research and Innovation at City College) magazine.  \octob/ 24th. \url{https://thericc.com/climate-chaos-and-ccny-a-new-professor-in-eas-dr-spencer-hill/} \\
     & Live on-air radio interview, KCBS San Francisco.  \jul/ 15th.\\
2022 & City College of New York Physics Club event with Harlem middle school students.  \nov/ 18th.\\
2021 & ``Ask Me Anything'' virtual Q+A session, ``AMA Session on Climate Action,'' Presidential Scholars Alumni Association.  \octob/ 28th.\\
     & Interview in \emph{The Medallion} (Newsletter of the Presidential Scholars Alumni Association).  ``Scholars in Climate Science: Observations from the Field.''  \sep/ 27th.\\
     & U.S. Presidential Scholars Foundation roundtable on human and planetary health, \jul/ 21st.\\
     & ``You Spin Me Right Round.''  Zoom-based presentation on planetary fluid flows for K-12 students, part of the EI Live K12 series, Earth Institute, Columbia University.  \feb/ 11th. \url{https://www.earth.columbia.edu/videos/view/you-spin-me-right-round}\\
2023- & Member, \emph{DIY}namics (organization advancing rotating tank-based geoscience teaching; \href{https://diynamics.github.io/}{https://diynamics.github.io}).\\
2017-23 & Co-founder and Co-Director, \emph{DIY}namics\\
2015- & Volunteer at many science outreach events at schools or open to the public\\

\end{longtable}
\vskip1em


\section*{SERVICE ACTIVITIES}
\subsection*{Professional}
\begin{longtable}{@{} L{\textwidth} @{}}

Associate Editor, \emph{Journal of the Atmospheric Sciences}, 2023-.\\

Review panelist, NSF GEO EMBRACE (Directorate for Geosciences EMpowering BRoader Academic Capacity and Education), \mar/ 2024.\\

Proposal referee: NSF Climate and Large-Scale Dynamics; NSF Cyberinfrastructure for Sustained Scientific Innovation; NSF EarthCube; NASA Juno Participating Scientist Program\\

Article referee, \emph{Nature}, \emph{Nature Climate Change}, \emph{Science Advances}, \emph{Nature Communications}, \emph{npj Climate and Atmospheric Science}, \emph{Journal of Advances in Modeling Earth Systems}, \emph{Journal of Climate}, \emph{Current Climate Change Reports}, \emph{Quarterly Journal of the Royal Meteorological Society}, \emph{Geophysical Research Letters}, \emph{Journal of the Atmospheric Sciences}, \emph{Surveys in Geophysics}, \emph{Paleoceanography and Paleoclimatology}, \emph{Geoscientific Model Development}, \emph{Climatic Change}, \emph{Climate Dynamics}, \emph{Journal of Geophysical Research - Atmospheres}, \emph{Weather and Climate Dynamics}, \emph{International Journal of Climatology}, \emph{Journal of Geoscience Education}, \emph{Atmospheric Science Letters}, \emph{Atmospheric Chemistry and Physics}\\

Advisory Board Member, ``Building Capacity for Investigating the Use of Spatial Reasoning in Fluid-Earth Science Disciplines,'' NSF Building Capacity in STEM Education Research (BCSER) program.  PI Peggy McNeal.  2023-2024.\\

Developer and maintainer, \texttt{aospy} open-source Python software package for automating analyses of climate datasets (\href{https://aospy.readthedocs.io}{https://aospy.readthedocs.io})\\

Code contributor, \texttt{xarray} package for labeled multidimensional arrays in Python\\
\end{longtable}

\subsection*{Departmental/University}
\begin{longtable}{@{} Y{0.08\textwidth} L{0.92\textwidth} @{}}
2024 & Presentation to new students, CCNY Division of Science Orientation event.  \aug/ 21st.\\
     & Organizer, CCNY EAS Machine Learning summer reading group\\
     & Presentation to undeclared students, CCNY Academic Awareness (``The Hub'') event.  \mar/ 19th.\\
     & Presentation to new students, CCNY Division of Science Orientation event.  \jan/ 23rd.\\
2015 & Organizer, Convection Journal Club, Program in Atmospheric and Oceanic Sciences, Princeton University\\
2013-15 & Organizer, Climate Sensitivity Journal Club, NOAA Geophysical Fluid Dynamics Laboratory\\
2012-13 & Organizer, Student/Postdoc Seminar Series, Program in Atmospheric and Oceanic Sciences, Princeton University\\
2012-13 & Representative to the faculty, Program in Atmospheric and Oceanic Sciences, Princeton University\\
\end{longtable}
\vskip1em

\section*{ADDITIONAL TRAINING}
\begin{longtable}{@{} Y{0.08\textwidth} L{0.92\textwidth} @{}}
2024 & ``Building Swift Meaningful Connections through Structured Conversations.''  Virtual workshop, Heterodox Academy, \octob/ 23rd\\ 
     & Workshop on Sustained Dialogue, CCNY, \aug/ 26th\\
     & ``Harnessing AI to Enhance Constructive Disagreement on Campus.'' Virtual workshop, Heterodox Academy, \aug/ 21st\\
     & ``AI and Pedagogy.'' Virtual workshop, Teaching and Learning Center, City College of New York, \apr/ 4th\\
2021 & Workshop on Inclusive and Culturally Competent Teaching Strategies in the Earth Sciences, Columbia University, \jan/ 8th\\
2020 & Racial Sensitivity Workshop, Columbia University Earth Institute, \jun/ 30th\\
2018 & 17th Swiss Climate Summer School, ``Earth System Variability Through Time,'' Grindelwald, Switzerland\\
2012-13 & Member, Princeton University Energy and Climate Scholars\\
2012 & NOAA Geophysical Fluid Dynamics Laboratory Summer School on Atmospheric Modeling\\
\end{longtable}
\vskip1em

\section*{PROFESSIONAL MEMBERSHIPS}
\begin{longtable}{@{} Y{0.06\textwidth} L{0.94\textwidth} @{}}
2010- & American Geophysical Union\\
2011- & American Meteorological Society\\
2021- & National Association of Geoscience Teachers\\
2024- & Heterodox Academy
\end{longtable}
\vskip1em

\end{document}

%%%%%
% Recommendations from \emph{The Professor Is In} by Karen Kelsky
% 1. General formatting rules
% \begin{itemize}
%   \item 1'' margins, 12-pt type, single-spaced
%   \item Candidate's name at top in 14-pt type
%   \item Headings in bold and all caps
%   \item Subheadings in bold only
%   \item No italics except for journal and book titles
%   \item One or two blank lines before each new heading
%   \item One blank line before each subheading
%   \item One blank line between each heading and its first entry
%   \item All elements left justified
%   \item No bullet points at all
%   \item No box or column formatting
%   \item The year (but not month or day) of every entry is left justified, with tabs or indents separating the year from the substance of the entry
%   \item No narrative verbiage anywhere
%   \item No description of ``duties'' under Teaching/Courses taught
%   \item No explanations of grants or fellowships
% \end{itemize}

% 2. Heading Material
% \begin{itemize}
%   \item Name at top, centered, in 14-pt type
%   \item (optional) The words ``Curriculum Vitae'' immediately underneath the name, centered, in 12-pt type, if appropriate to your field
%   \item (optional) The date, immediately below, centered.  Senior scholars always date their CVs.
%   \item Below the date, your institutional and your home addresses, pulus phone numbers, should be on the left and the right side of the page, respectively, parallel to one another.
% \end{itemize}

% 3. Content
% \begin{description}
%   \item[Education] Always first.  List by degree, not by institution.  List Ph.D., M.A., B.S., in descneding order.  Give department, institution, and year of completion.  Do not include advisor(s) name(s) if you're beyond one year past your defense.
%   \item[Professional Appointments] Postdoctoral positions go here.  Give the institution, department, title, and dates (year only) of employment.
%   \item[Publications] Use subheadings as appropriate for your field and record: Books, Edited Volumes, Refereed Journal Articles, Book Chapters, Conference Proceedings, Encyclopedia Entries, Book Reviews, Manuscripts in Submission (Give journal title.), Manuscripts in Preparation, Other Publications.  Forthcoming accepted publications are listed w/ published pieces listed at the very top and either ``in press'' or ``accepted'' in place of the year.
%   \item[Awards and Honors] Give name of award and institutional location.  Year at left.  Reverse chronological order.
%   \item[Grants and Fellowships] Give funder, institutional location in which received/used, and year.
%   \item[Invited Talks] Thes are talks to which you have been invited at other campuses, not your own.  Give title, institutional location, and date (year only) at left.
%   \item[Conference Activity/Participation] Use subheadings as appropriate: Conferences/Symposia Organized, Panels Organized, Papers Presented, Discussant.  Entries will include name of paper, name of conference, and date (year only) on left.  Month and date range of conference goes in the entry itself.  Future conferences should be listed here, once you have had a paper officially accepted but not before.  No need to use ``forthcoming'' for future conferences; the dates will be in the future and thus will be the first dates listed.
%   \item[Campus or Departmental Talks] Talks that you were asked to give in your own department or on your own campus.  List as you would invited talks.
%   \item[Teaching Experience] Subdivide either by institutional location, area/field, graduate/undergraduate, or some combination of these as appropriate to your particular case.  TA experience goes here.  No narrative verbiage under any course title.
%   \item[Service to Profession] Include journal manuscript review work, with journal titles (manuscript review can be given its own separate heading if you do a lot of this work).  It is not typical to highlight the timing of service work, so the convention is that the entry itself is often left justified, with the year in the entry, deviating from the year-to-the-left rule.
%   \item[Departmental/University Service] Include search committees and other committee work, appointments to faculty senate, and so on.
%   \item[Community Involvement/Outreach] (optional)
%   \item[Media Coverage] (optional)
%   \item[Professional Memberships or Affiliations] List vertically all professional organizations of which you are a member.  Include years of joining when you are more senior---it demonstrates a length of commitment to a field.
%     \item[References] List references vertically.  Give name and full title.  Give full snail mail contact information, telephone, and email.  Do not give narrative verbiage or explanation.  The only exception is a single reference that may be identified as a teaching reference.  This would be the fourth of four references.
% \end{description}

% Central organizing principle of the CV: the Principle of Peer Review.  Things that are peer-reviewed and competitive take precedence over things that are not.
%%%%%

%%% Local Variables:
%%% mode: latex
%%% TeX-master: t
%%% TeX-engine: xetex
%%% End:
