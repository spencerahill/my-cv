% Intended LaTeX compiler: xelatex
% Use my "shillcv.cls" file.
\documentclass[letterpaper,11pt]{shillcv}

% Use my shortcuts for commonly used text.
\usepackage{abbrevs}

% Recommendations from \emph{The Professor Is In} by Karen Kelsky
% 1. General formatting rules
% \begin{itemize}
%   \item 1'' margins, 12-pt type, single-spaced
%   \item Candidate's name at top in 14-pt type
%   \item Headings in bold and all caps
%   \item Subheadings in bold only
%   \item No italics except for journal and book titles
%   \item One or two blank lines before each new heading
%   \item One blank line before each subheading
%   \item One blank line between each heading and its first entry
%   \item All elements left justified
%   \item No bullet points at all
%   \item No box or column formatting
%   \item The year (but not month or day) of every entry is left justified, with tabs or indents separating the year from the substance of the entry
%   \item No narrative verbiage anywhere
%   \item No description of ``duties'' under Teaching/Courses taught
%   \item No explanations of grants or fellowships
% \end{itemize}

% 2. Heading Material
% \begin{itemize}
%   \item Name at top, centered, in 14-pt type
%   \item (optional) The words ``Curriculum Vitae'' immediately underneath the name, centered, in 12-pt type, if appropriate to your field
%   \item (optional) The date, immediately below, centered.  Senior scholars always date their CVs.
%   \item Below the date, your institutional and your home addresses, pulus phone numbers, should be on the left and the right side of the page, respectively, parallel to one another.
% \end{itemize}

% 3. Content
% \begin{description}
%   \item[Education] Always first.  List by degree, not by institution.  List Ph.D., M.A., B.S., in descneding order.  Give department, institution, and year of completion.  Do not include advisor(s) name(s) if you're beyond one year past your defense.
%   \item[Professional Appointments] Postdoctoral positions go here.  Give the institution, department, title, and dates (year only) of employment.
%   \item[Publications] Use subheadings as appropriate for your field and record: Books, Edited Volumes, Refereed Journal Articles, Book Chapters, Conference Proceedings, Encyclopedia Entries, Book Reviews, Manuscripts in Submission (Give journal title.), Manuscripts in Preparation, Other Publications.  Forthcoming accepted publications are listed w/ published pieces listed at the very top and either ``in press'' or ``accepted'' in place of the year.
%   \item[Awards and Honors] Give name of award and institutional location.  Year at left.  Reverse chronological order.
%   \item[Grants and Fellowships] Give funder, institutional location in which received/used, and year.
%   \item[Invited Talks] Thes are talks to which you have been invited at other campuses, not your own.  Give title, institutional location, and date (year only) at left.
%   \item[Conference Activity/Participation] Use subheadings as appropriate: Conferences/Symposia Organized, Panels Organized, Papers Presented, Discussant.  Entries will include name of paper, name of conference, and date (year only) on left.  Month and date range of conference goes in the entry itself.  Future conferences should be listed here, once you have had a paper officially accepted but not before.  No need to use ``forthcoming'' for future conferences; the dates will be in the future and thus will be the first dates listed.
%   \item[Campus or Departmental Talks] Talks that you were asked to give in your own department or on your own campus.  List as you would invited talks.
%   \item[Teaching Experience] Subdivide either by institutional location, area/field, graduate/undergraduate, or some combination of these as appropriate to your particular case.  TA experience goes here.  No narrative verbiage under any course title.
%   \item[Service to Profession] Include journal manuscript review work, with journal titles (manuscript review can be given its own separate heading if you do a lot of this work).  It is not typical to highlight the timing of service work, so the convention is that the entry itself is often left justified, with the year in the entry, deviating from the year-to-the-left rule.
%   \item[Departmental/University Service] Include search committees and other committee work, appointments to faculty senate, and so on.
%   \item[Community Involvement/Outreach] (optional)
%   \item[Media Coverage] (optional)
%   \item[Professional Memberships or Affiliations] List vertically all professional organizations of which you are a member.  Include years of joining when you are more senior---it demonstrates a length of commitment to a field.
%     \item[References] List references vertically.  Give name and full title.  Give full snail mail contact information, telephone, and email.  Do not give narrative verbiage or explanation.  The only exception is a single reference that may be identified as a teaching reference.  This would be the fourth of four references.
% \end{description}

% Central organizing principle of the CV: the Principle of Peer Review.  Things that are peer-reviewed and competitive take precedence over things that are not.

\title{Spencer Alan Hill}

\begin{document}

\maketitle
\thispagestyle{CVfooter}

\begin{tabular}{@{} L{0.4\textwidth} L{0.1\textwidth} L{0.4\textwidth} @{}}
207B Oceanography & & \href{mailto:shill@ldeo.columbia.edu}{shill@ldeo.columbia.edu} \\
Lamont-Doherty Earth Observatory & & \href{https://www.ldeo.columbia.edu/~shill}{https://www.ldeo.columbia.edu/\~{}shill}\\
61 Route 9W & & (913) 515-8527\\
Palisades, NY 10964 & &\\
\end{tabular}

\vskip1em
\hrule
\vspace{0.5cm}

\section*{EDUCATION}
\begin{longtable}{@{} Y{0.05\textwidth} p{0.05\textwidth} p{0.6\textwidth} p{0.3\textwidth} @{}}
2016 & Ph.D. & Atmospheric and Oceanic Sciences & Princeton University\\
2011 & B.S.  & Atmospheric and Oceanic Sciences; Applied Mathematics & UCLA\\
\end{longtable}
\vskip1em


\section*{PROFESSIONAL APPOINTMENTS}
\begin{longtable}{@{} Y{0.08\textwidth} L{0.92\textwidth} @{}}
2021- & Associate Research Scholar, Program in Atmospheric and Oceanic Sciences, Princeton University\\
2021- & Adjunct Associate Research Scientist, Lamont-Doherty Earth Observatory, Columbia University\\
2021 & Associate Research Scientist, Lamont-Doherty Earth Observatory, Columbia University\\
2019-21 & Postdoctoral Research Scientist, Lamont-Doherty Earth Observatory, Columbia University\\
2016-19 & Postdoctoral Research Scientist, dual appointment, Division of Geological and Planetary \mbox{Sciences}, California Institute of Technology, and \dept/ of Atmospheric and Oceanic \mbox{Sciences}, UCLA\\
2011-16 & Graduate Research Assistant, Program in Atmospheric and Oceanic Sciences, Princeton \mbox{University}
\end{longtable}
\vskip1em


\section*{GRANT PROPOSALS SELECTED FOR FUNDING AS PRINCIPAL INVESTIGATOR}
\begin{longtable}{@{} Y{0.05\textwidth} L{0.95\textwidth} @{}}
2021 & ``How do energy fluxes link precipitation variability across the tropical weather-climate continuum?'' NSF Climate and Large-Scale Dynamics.  (to be awarded pending completion of NSF internal administrative review)\\
\end{longtable}
\vskip1em


\section*{AWARDS AND FELLOWSHIPS}
\begin{longtable}{@{}  Y{0.05\textwidth} L{0.95\textwidth} @{}}
2021 & Columbia University Nominee, Blavatnik Regional Award for Young Scientists\\
2019 & Columbia University Earth Institute Postdoctoral Research Fellowship (2019-21)\\
2016 & California Institute of Technology Foster and Coco Stanback Postdoctoral Research Fellowship (deferred to 2018-19)\\
     & NSF Atmospheric and Geospace Sciences Postdoctoral Research Fellowship (2016-18)\\
2013 & \dept/ of Defense National Defense Science and Engineering Graduate Research Fellowship (2013-16)\\
2012 & Princeton University Elliotte Robinson Little '25 Fellowship\\
2011 & American Meteorological Society Annual Meeting Climate Change Travel Scholarship\\
     & NSF Graduate Research Fellowship Honorable Mention\\
     & UCLA Magna Cum Laude and College Honors graduation distinctions\\
2009 & National Oceanic and Atmospheric Administration Ernest F. Hollings Undergraduate Scholarship\\
2007 & United States Presidential Scholar, conferred by the U.S. \dept/ of Education Commission on Presidential Scholars
\end{longtable}
\vskip1em


\section*{PUBLICATIONS}
\subsection*{Articles submitted/under review/in revision}
\begin{longtable}{@{}  p{0.04\textwidth} @{} L{0.96\textwidth} @{}}
16. & \textbf{Hill SA}, AH Sobel, M Biasutti, and MA Cane.  ``On the all-India rainfall index and sub-India rainfall heterogeneity.''  In revision, \emph{Geophys. Res. Lett.}  \href{https://arxiv.org/abs/2104.12306}{https://arxiv.org/abs/2104.12306}.\\
15. & Biasutti M, \textbf{SA Hill}, and A Voigt.  ``The Effect Of An Equatorial Continent On The Tropical Rain Belt. Part 2: Summer Monsoons.''  In revision, \emph{J. Climate}.   \href{https://doi.org/10.1002/essoar.10508669.1}{https://doi.org/10.1002/essoar.10508669.1}.\\
14. & \textbf{Hill SA}, NJ Burls, A Fedorov, and TM Merlis.  ``Symmetric and antisymmetric components of polar-amplified warming.''  Revised for \emph{J. Climate}.  \href{http://arxiv.org/abs/2012.09228}{http://arxiv.org/abs/2012.09228}.\\
\end{longtable}

\subsection*{Refereed Journal Articles}
\begin{longtable}{@{} Y{0.05\textwidth} >{\color{black}} p{0.04\textwidth} @{} L{0.91\textwidth} @{}}
2021 & 13. & \textbf{Hill SA}, S Bordoni, and JL Mitchell.  ``Solsticial Hadley Cell ascending edge theory from supercriticality.''  \emph{J. Atmos. Sci}, \textbf{78}, 1999-2011.  doi: \href{https://doi.org/10.1175/JAS-D-20-0341.1}{10.1175/JAS-D-20-0341.1}.\\
     & 12. & Mitchell, JL and \textbf{SA Hill}.  ``Constraints from invariant
subtropical vertical velocities on the scalings of Hadley cell strength and
downdraft width with rotation rate.''  \emph{J. Atmos. Sci}, \textbf{78}, 1445-1463.  doi: \href{https://doi.org/10.1175/JAS-D-20-0191.1}{10.1175/JAS-D-20-0191.1}\\
2020 &  11. & \textbf{Hill SA}, S Bordoni, and JL Mitchell.
``Axisymmetric Hadley Cell theory with a fixed tropopause temperature rather
than height.'' \emph{J. Atmos. Sci.}, \textbf{77}, 1279-1294.  doi: \href{https://doi.org/10.1175/JAS-D-19-0169.1}{10.1175/JAS-D-19-0169.1}.\\
2019 & 10. & \textbf{Hill SA}.  ``Theories for past and future monsoon rainfall
changes.'' \emph{Curr. Clim. Change Rep.}, \textbf{5}, 160-171.  doi: \href{https://doi.org/10.1007\%2Fs40641-019-00137-8}{10.1007\%2Fs40641-019-00137-8}.\\
     & 9. & \textbf{Hill SA}, S Bordoni, and JL Mitchell.
``Axisymmetric constraints on cross-equatorial Hadley cell extent.''
\emph{J. Atmos. Sci.}, \textbf{76}, 1547-1564.  doi: \href{https://doi.org/10.1175/JAS-D-18-0306.1}{10.1175/JAS-D-18-0306.1}.\\
2018 & 8. & \textbf{Hill SA}, JM Lora, N Khoo, SP Faulk, and
JM Aurnou.  ``Affordable rotating fluid demonstrations for
geoscience education: The \emph{DIYnamics} project.''  \emph{Bull.
Am. Met. Soc.}, \textbf{99}, 2529-2538.  doi: \href{https://doi.org/10.1175/BAMS-D-17-0215.1}{10.1175/BAMS-D-17-0215.1}.\\
     & 7. & \textbf{Hill SA}, Y Ming, and M Zhao.  ``Robust responses of the
Sahelian hydrological cycle to global warming.''  \emph{J. Climate}, \textbf{31}, 9793-9814.  doi: \href{https://doi.org/10.1175/JCLI-D-18-0238.1}{10.1175/JCLI-D-18-0238.1}.\\
     & 6. & Smyth J, \textbf{SA Hill}, and Y Ming.  ``Simulated responses of
the West African monsoon and zonal-mean tropical precipitation to early
Holocene orbital forcing.''  \emph{Geophys. Res. Lett.}, \textbf{45},
12,049-12,057.  doi: \href{https://doi.org/10.1029/2018GL080494}{10.1029/2018GL080494}.\\
2017 & 5. & \textbf{Hill SA}, Y Ming, IM Held, and M Zhao.  ``A moist
static energy budget-based analysis of the Sahel rainfall response to uniform
oceanic warming.''  \emph{J. Climate}, \textbf{30}, 5637-5660.  doi: \href{https://doi.org/10.1175/JCLI-D-16-0785.1}{10.1175/JCLI-D-16-0785.1}.\\
     & 4. & Brown PT, Y Ming, W Li, and \textbf{SA Hill}.  ``Change
in the magnitude and mechanisms of unforced low-frequency surface temperature
variability in a warmer climate.''  \emph{Nature Climate Change}, \textbf{7}, 743-748.  \href{https://doi.org/10.1038/nclimate3381}{10.1038/nclimate3381}.\\
     & 3. & Jeevanjee N, P Hassanzadeh, \textbf{SA Hill}, and A Sheshadri.  "A perspective on climate model hierarchies."  \emph{J.  Adv. in Mod. Earth Sys.}, \textbf{9}, 1760-1771.  doi: \href{https://doi.org/10.1002/2017MS001038}{10.1002/2017MS001038}.\\
2015 & 2. & \textbf{Hill SA}, Y Ming, and IM Held.  ``Mechanisms of forced
tropical meridional energy flux change.''  \emph{J. Climate}, \textbf{28}, 1725-1742.  doi: \href{http://dx.doi.org/10.1175/JCLI-D-14-00165.1}{10.1175/JCLI-D-14-00165.1}.\\
& & \hspace{1cm} Corrigendum: \href{https://dx.doi.org/10.1175/JCLI-D-16-0485.1}{https://dx.doi.org/10.1175/JCLI-D-16-0485.1}.\\
2012 & 1. & \textbf{Hill SA} and Y Ming.  ``Nonlinear climate response to regional
brightening of tropical marine stratocumulus.''  \emph{Geophys. Res. Lett.},
\textbf{39}, L15707, 5 pp.  doi: \href{http://dx.doi.org/10.1029/2012GL052064}{10.1029/2012GL052064}.\\
\end{longtable}


\subsection*{Articles in preparation}
\begin{longtable}{@{} L{\textwidth} @{}}
\textbf{Hill SA}, AH Sobel, M Biasutti, and MA Cane.  ``Timescales of the Equatorial Indian Ocean Oscillation and implications for seasonal prediction.''\\
\textbf{Hill SA, S Bordoni, and JL Mitchell}.  ``Toward a unified theory for the Hadley cell descending and ascending edges throughout the annual cycle.''\\
Zamir Meyers D and \textbf{SA Hill}. ``Indian summer monsoon extreme rainfall interannual variability.''\\

\end{longtable}

\subsection*{Book reviews}
\begin{longtable}{@{}  Y{0.05\textwidth} L{0.95\textwidth} @{}}
2012 & \textbf{Hill SA}  ``A head in the clouds elucidates climate'' (book
review of \emph{Atmosphere, Clouds, and Climate} by David Randall).  \emph{Science}, \textbf{337},
1 pp., doi: \href{http://dx.doi.org/10.1126/science.1225615}{10.1126/science.1225615}.\\
\end{longtable}

\subsection*{Other publications}
\begin{longtable}{@{}  Y{0.08\textwidth} L{0.92\textwidth} @{}}
2017-20 &  \raggedright Numerous blog posts for the DIYnamics blog.  Available at \href{https://diynamics.github.io/blog/author/spencer-hill.html}{https://diynamics.github.io/blog/author/spencer-hill.html}.\\

% SAH note: this whole blog appears to have gone offline.
% So it's not appropriate to include anymore.
% 2017 & \textbf{Hill, Spencer A.} and Spencer K. Clark.  ``What’s needed for the Future
% of AOS Python?  Tools for Automating AOS Data Analysis and Management.''
% Invited guest blog post on ``PyAOS'' blog.
% \href{http://pyaos.johnny-lin.com/?p=1546}{http://pyaos.johnny-lin.com/?p=1546}.\\

\end{longtable}


\section*{INVITED COLLOQUIA AND SEMINARS}
\begin{longtable}{@{} Y{0.05\textwidth} L{0.95\textwidth} @{}}
2021 & Atmosphere, Ocean, and Climate Dynamics seminar series, Yale University, \feb/ 25th.\\
     & Meteorology Seminar Series, \dept/ of Earth, Atmospheric, and Oceanic Sciences, Florida State University, \feb/ 18th.\\
2019 & Earth Research Institute Monthly Climate Meeting, University of California -- Santa Barbara, \jan/ 7th.\\
2018 & Division of Ocean and Climate Physics Seminar Series, Lamont-Doherty Earth Observatory, \feb/ 16th.\\
     & NOAA Geophysical Fluid Dynamics Laboratory informal seminar series, \feb/ 14th.\\
2016 & Center for Atmosphere Ocean Science, New York University, \nov/ 9th.\\
     & Gaede Institute for the Liberal Arts, Natural and Behavioral Sciences Lecture, Westmont College, \octob/ 13th.\\
2015 & New York University, Courant Institute of Mathematical Sciences, Center for Atmospheric and Oceanic Sciences, Student seminar series, New York, NY\\
2014 & \dept/ of Geophysics, Yale University, \octob/ 9th.\\
\end{longtable}
\vskip1em


\section*{CONFERENCE ACTIVITIES}
\subsection*{Chaired sessions}
\begin{longtable}{@{} Y{0.05\textwidth} L{0.95\textwidth} @{}}
2018 & ``Monsoons: Observations, Subseasonal, Seasonal, and Interannual to Decadal Variability, Forecast, Climate Change, and Extremes III.''  AGU Fall Meeting, Washington, D.C., \dec/ 11th. \\
2017 & ``Idealized approaches to the atmospheric and oceanic circulation II.'' American Meteorological Society 21st Conference on Atmospheric and Oceanic Fluid Dynamics, Portland, OR, \jun/ 26th.\\
2016 & ``Tropical circulations and their sensitivities to changes in climate I.''  AGU Fall Meeting, San Francisco, CA, \dec/ 16th.\\
     & ``Tropical convection and radiative convective equilibrium.''  World Climate Research Programme Model Hierarchies Workshop, Princeton, NJ, \nov/ 3rd.\\
\end{longtable}

\subsection*{Invited Conference Talks}
\begin{longtable}{@{} Y{0.05\textwidth} L{0.95\textwidth} @{}}
2020 & European Geophysical Union Annual Meeting (cancelled due to COVID-19).\\
     & Continental Climate Change Workshop (postponed indefinitely due to COVID-19).\\
2019 & ``Toward an analytical, predictive theory for the location of Hadley and monsoonal cell ascending branches.''  AGU Fall Meeting, San Francisco, CA, \dec/ 10th.\\
2016 & ``\texttt{infinite-diff} and \texttt{animal-spharm}: \texttt{xarray}-based finite differencing and spherical harmonics.''  Columbia University Python and Atmospheric and Oceanic Sciences Workshop, New York, NY, \nov/ 12th.\\
2015 & ``Towards constraining Sahel rainfall responses to global mean temperature changes.''   Linde Center for Global Environmental Science, California Institute of Technology, Monsoons: Past, Present and Future workshop, Pasadena, CA, \may/ 21st.\\
\end{longtable}

\subsection*{Other Conference Presentations}
\begin{longtable}{@{} Y{0.05\textwidth} L{0.95\textwidth} @{}}
2021 & ``Connecting sub-India, sub-seasonal monsoon rainfall variability with all-India, all-summer monsoon rainfall.'' AMS Annual Meeting (virtual).  Talk.  \jan/ 14th.\\
2020 & ``Sub-India summer monsoon rainfall variability and its implications for all-India summer monsoon rainfall prediction.''  AGU Fall Meeting (virtual).  Poster.  \dec/ 1st-17th.\\
2019 & ``Simulated polar amplification and its causes on decadal to millennial timescales.''  Poster.  AGU Fall Meeting.  San Francisco, CA.  \dec/ 10th.\\
     & ``Modernizing Axisymmetric Hadley Cell and Monsoon Theory.''  Talk.  AMS 22nd Conference on Atmospheric and Oceanic Fluid Dynamics.  Portland, ME.  \jun/ 25th.\\
2018 & ``What sets the locations of the solsticial cross-equatorial Hadley cell edges?''  Talk.  AGU Fall Meeting.  Washington, DC.  \dec/ 13th.\\
     & ``Towards transient simulation of the Green Sahara onset and demise through idealized modeling of vegetation-land-atmosphere interactions.''  Poster.  17th Swiss Climate Summer School: Earth system variability through time.  Grindelwald, Switzerland.  \aug/ 28th.\\
     & ``What Determines the ITCZ Position During Solsticial Seasons on Earth and Other Planets?''  Talk.  AMS 33rd Conference on Hurricanes and Tropical Meteorology.  Ponte Vedra, FL.  \apr/ 16th.\\
2017 & ``Dry Rainbelts: Understanding Boundary Layer Controls on the ITCZ Using a Dry Dynamical Core.''  Talk.  AGU Fall Meeting.  New Orleans, LA.  \dec/ 14th.\\
     & ``Towards transient simulation of the Green Sahara onset and demise through idealized modeling of vegetation-land-atmosphere interactions.''  Poster.  Gordon Research Conference on Radiation and Climate.  Bates College, Lewiston, ME.  \jul/ 19th.\\
     & ``Control of convergence zone migrations by planetary parameters.''  Poster.  AMS 21st Conference on Atmospheric and Oceanic Fluid Dynamics.  Portland, OR.  \jun/ 27th.\\
     & ``Automate your climate and weather data analysis with aospy.''  Talk.  AMS Annual Meeting, Seattle, WA.  \jan/ 24th.\\
     & ``Energetic and precipitation responses in the Sahel to sea surface temperature perturbations.''  Talk.  AMS Annual Meeting, Seattle, WA.  \jan/ 24th.\\
2016 & ``Robust drying influence of mean ocean surface warming on The Sahel and implications for constraining future rainfall change.''  Poster.  AGU Fall Meeting, San Francisco, CA.  \dec/ 16th.\\

     & ``A hierarchy of perturbation complexites: Case study of Sahel rainfall response to global warming''  Poster.  WCRP Model Hierarchies Workshop, Princeton University, Princeton, NJ.  \nov/ 2nd.\\
2015 & ``Towards constraining future rainfall in the Sahel using the moist static energy budget.'' Talk.  AGU Fall Meeting, San Francisco, CA.  \dec/ 14th.\\
     & ``Convection scheme, cloud, and stability effects on Sahel rainfall response to uniform warming.''  Poster.  AMS Annual Meeting, Phoenix, AZ. \jan/ 6th.\\
2014 & ``Convection scheme, cloud, and stability effects on Sahel rainfall response to uniform warming.''  Poster.  AGU Fall Meeting, San Francisco, CA.  \dec/ 15th.\\
     & ``Mechanisms of forced tropical meridional energy flux change.''  Poster.  Latsis Symposium, ETH Zurich, Zurich, Switzerland. \jun/ 19th.\\
     & ``Mean and extreme tropical precipitation changes caused by the uniform and spatially varying components of anthropogenic forcing.''  Talk.  AMS Annual Meeting, Atlanta, GA.  \feb/ 5th.\\
2013 & ``Mechanisms of forced tropical meridional energy flux change.''  Talk.  AGU Fall Meeting, San Francisco, CA. \dec/ 13th.\\
     & ``Mechanisms of forced tropical meridional energy flux change.''  Talk.  Graduate Climate Conference, Woods Hole Oceanographic Institution, Woods Hole, MA.  \nov/ 2nd.\\
     & ``Mechanisms of forced tropical meridional energy flux change.'' Poster presentation.  Gordon Research Conference, Colby-Sawyer College, New London, NH.  \jul/ 9th.\\
2012 & ``Climate response to a geoengineered brightening of subtropical marine boundary clouds.''  Poster.  11th Annual Student Conference at the AMS Annual Meeting, New Orleans, LA.  \jan/ 22nd.\\
2010 & ``Climate response to a geoengineered brightening of subtropical marine boundary clouds.''  Poster.  San Francisco, CA.  \dec/ 14th.\\
     & ``Climate response to a geoengineered brightening of subtropical marine boundary clouds.'' Talk.  Special Symposium on Aerosols in Geoengineering at the American Association for Aerosol Research 29th Annual Conference.  Portland, OR.  \octob/ 26th.\\
     & ``Investigating climate response to geoengineering using a global climate model.''  Talk.  National Oceanic and Atmospheric Administration Office of Education Science Symposium, Silver Spring, MD.  \aug/ 3rd.\\
\end{longtable}
\vskip1em


\section*{CAMPUS AND DEPARTMENTAL TALKS}
\begin{longtable}{@{} Y{0.05\textwidth} L{0.95\textwidth} @{}}
2019 & Lamont-Doherty Earth Observatory Postdoc Symposium (poster).  \sep/ 11th.\\
2018 & \dept/ of Atmospheric and Oceanic Sciences, UCLA.  \nov/ 7th.\\
2016 & Division of Geological and Planetary Sciences, California Institute of Technology.  \octob/ 26th.\\
     & \dept/ of Atmospheric and Oceanic Sciences, UCLA.  \octob/ 5th.\\
2015 & Dynamics Seminar Series, Program in Atmospheric and Oceanic Sciences, Princeton University.  \mar/ 13th.\\
2012 & Student/Postdoc Seminar Series, Program in Atmospheric and Oceanic Sciences, Princeton University.  \feb/ 28th.\\
2011 & Graduate Research Symposium, \dept/ of Geosciences, Princeton University. \nov/ 11th.\\
\end{longtable}
\vskip1em


\section*{UNDERGRADUATE RESEARCH ADVISING}
\subsection*{As primary advisor}
\begin{longtable}{@{}  Y{0.05\textwidth} L{0.95\textwidth} @{}}
2021 & Destiny Zamir Meyers, Columbia University\\
     & Matthew Donahue, Columbia University\\
2020 & Valentina Rojas-Posada, Barnard University (partial funding awarded by competition from Columbia University Earth Institute)\\
2017 & Norris Khoo, UCLA\\
     & Micah Kim, UCLA\\
2016 & Juliet Olsen, UCLA\\
\end{longtable}

\subsection*{Summer interns, as co-advisor}
\begin{longtable}{@{}  Y{0.05\textwidth} L{0.95\textwidth} @{}}
2015 & Jane Smyth, NOAA Geophysical Fluid Dynamics Laboratory\\
2014 & Marjahn Finlayson, NOAA Geophysical Fluid Dynamics Laboratory\\
2013 & Colin Raymond, NOAA Geophysical Fluid Dynamics Laboratory\\
\end{longtable}


\section*{TEACHING ACTIVITIES}
\subsection*{Teaching Assistant}
\begin{longtable}{@{} Y{0.05\textwidth} L{0.95\textwidth} @{}}
2014 & ``Physics of Earth: The Habitable Planet.''  Upper-division undergraduate course.  \dept/ of Geosciences, Princeton University.\\
\end{longtable}

\subsection*{Certifications}
\begin{longtable}{@{} Y{0.05\textwidth} L{0.95\textwidth} @{}}
2016 & Teaching Transcript Certification, McGraw Center for Teaching and Learning, Princeton University.\\
\end{longtable}
\vskip1em

\subsection*{Guest Lectures}
\begin{longtable}{@{} Y{0.05\textwidth} L{0.95\textwidth} @{}}
2018 & Three lectures for ``Introduction to Atmospheric and Oceanic Fluids,'' graduate-level course, \dept/ of Atmospheric and Oceanic Sciences, UCLA.\\
& \hspace{1cm}``The general circulation of the atmosphere: Energetics.''  \nov/ 13th.\\
& \hspace{1cm}``The general circulation of the atmosphere: Momentum.''  \nov/ 15th.\\
& \hspace{1cm}``The general circulation of the atmosphere: Lab.''  \nov/ 29th.\\
2014 & Two lectures for ``Physics of Earth: The Habitable Planet,'' upper-division undergraduate course, \dept/ of Geosciences, Princeton University.\\
& \hspace{1cm}``Climate modeling and climate change projections''\\
& \hspace{1cm}``Weather: From orbital order, atmospheric chaos''\\
\end{longtable}
\vskip1em


\section*{SCIENCE OUTREACH AND MEDIA}
\begin{longtable}{@{} Y{0.05\textwidth} L{\textwidth} @{}}
2021 & ``Ask Me Anything'' virtual Q+A session, ``AMA Session on Climate Action,'' Presidential Scholars Alumni Association.  \octob/ 28th.\\
     & Interview in \emph{The Medallion} (Newsletter of the Presidential Scholars Alumni Association).  ``Scholars in Climate Science: Observations from the Field.''  \sep/ 27th.\\
     & U.S. Presidential Scholars Foundation roundtable on human and planetary health, \jul/ 21st.\\
     & ``You Spin Me Right Round.''  Zoom-based presentation on planetary fluid flows for K-12 students, part of the EI Live K12 series, Earth Institute, Columbia University.  \feb/ 11th. \url{https://www.earth.columbia.edu/videos/view/you-spin-me-right-round}\\
2017- & Co-founder and Co-Director, \emph{DIY}namics (organization advancing rotating tank-based geoscience teaching; \href{https://diynamics.github.io/}{https://diynamics.github.io}).\\
2015- & Volunteer at 10+ science outreach events either open to the public or presented at middle schools.\\

\end{longtable}
\vskip1em


\section*{SERVICE ACTIVITIES}
\subsection*{Professional}
\begin{longtable}{@{} L{\textwidth} @{}}

Review Editor, Editorial Board of Predictions and Projections, \emph{Frontiers in Climate}\\

Proposal referee, NSF Climate and Large-Scale Dynamics, NSF EarthCube, NASA Juno Participating Scientist Program\\

Article referee, \emph{Nature Climate Change}, \emph{Nature Communications}, \emph{npj Climate and Atmospheric Science}, \emph{Journal of Advances in Modeling Earth Systems}, \emph{Journal of Climate}, \emph{Current Climate Change Reports}, \emph{Quarterly Journal of the Royal Meteorological Society}, \emph{Geophysical Research Letters}, \emph{Journal of the Atmospheric Sciences}, \emph{Geoscientific Model Development}, \emph{Climatic Change}, \emph{Climate Dynamics}, \emph{Journal of Geophysical Research - Atmospheres}\\

Developer and maintainer, \texttt{aospy} open-source Python software package for automating analyses of climate datasets (\href{https://aospy.readthedocs.io}{https://aospy.readthedocs.io})\\

Code contributor, \texttt{xarray} package for labeled multidimensional arrays in Python\\
\end{longtable}

\subsection*{Departmental/University}
\begin{longtable}{@{} Y{0.08\textwidth} L{0.92\textwidth} @{}}

2015 & Organizer, Convection Journal Club, Program in Atmospheric and Oceanic Sciences, Princeton University\\

2013-15 & Organizer, Climate Sensitivity Journal Club, NOAA Geophysical Fluid Dynamics Laboratory\\

2012-13 & Organizer, Student/Postdoc Seminar Series, Program in Atmospheric and Oceanic Sciences, Princeton University\\

2012-13 & Representative to the faculty, Program in Atmospheric and Oceanic Sciences, Princeton University\\
\end{longtable}
\vskip1em


\section*{ADDITIONAL TRAINING}
\begin{longtable}{@{} Y{0.08\textwidth} L{0.92\textwidth} @{}}
2021 & Workshop on Inclusive and Culturally Competent Teaching Strategies in the Earth Sciences, Columbia University, \jan/ 8th\\
2020 & Racial Sensitivity Workshop, Columbia University Earth Institute, \jun/ 30th\\
2018 & 17th Swiss Climate Summer School, ``Earth System Variability Through Time,'' Grindelwald, Switzerland\\
2012-13 & Member, Princeton University Energy and Climate Scholars\\
2012 & NOAA Geophysical Fluid Dynamics Laboratory Summer School on Atmospheric Modeling\\
\end{longtable}


\section*{PROFESSIONAL MEMBERSHIPS}
\begin{longtable}{@{} Y{0.06\textwidth} L{0.94\textwidth} @{}}
2010- & American Geophysical Union\\
2011- & American Meteorological Society\\
2021- & National Association of Geoscience Teachers\\
\end{longtable}
\vskip1em

\end{document}
%%% Local Variables:
%%% mode: latex
%%% TeX-master: t
%%% TeX-engine: xetex
%%% End:
