% Intended LaTeX compiler: pdflatex
\documentclass[12pt,letterpaper]{shillcv}
\usepackage[utf8]{inputenc}
\usepackage[T1]{fontenc}
\usepackage{graphicx}
\usepackage{grffile}
\usepackage{longtable}
\usepackage{wrapfig}
\usepackage{rotating}
\usepackage[normalem]{ulem}
\usepackage{amsmath}
\usepackage{textcomp}
\usepackage{amssymb}
\usepackage{capt-of}
\usepackage[colorlinks=true,linkcolor=black,citecolor=black]{hyperref}
\author{Spencer Alan Hill, Ph.D.}
\date{}
\title{Postdoctoral Research Scientist\\\medskip
\large Columbia University | Lamont-Doherty Earth Observatory}
\begin{document}

\maketitle
\section*{Contact information}
\label{sec:org2974eb8}
\href{mailto:shill@ldeo.columbia.edu}{shill@ldeo.columbia.edu}\\
\url{https://www.ldeo.columbia.edu/\~shill}\\

207B Oceanography\\
Lamont-Doherty Earth Observatory\\
61 Route 9W\\
Palisades, NY 10964\\

ORCID: \href{http://orcid.org/0000-0001-8672-0671}{0000-0001-8672-0671}\\
Twitter: \href{https://twitter.com/spencerahill}{@spencerahill}\\
Github: \url{https://github.com/spencerahill}
\section*{Employment history}
\label{sec:org551cb18}
\subsection*{Current position}
\label{sec:orgaffce14}
Earth Institute Research Fellow, Columbia University
Lamont-Doherty Earth Observatory.  Advisors Dr. Michela Biasutti and Dr. Adam
Sobel.  August 2019-August 2021.

\subsection*{Past positions}
\label{sec:orgd056e2f}
\begin{itemize}
\item Foster and Coco Stanback Postdoctoral Research Fellow, California Institute of
Technology Division of Geological and Planetary Sciences (GPS), and Visiting
Assistant Project Scientist, UCLA Dept. of Earth, Planetary and Space Sciences
(EPSS).  Advisors Prof. Simona Bordoni (Caltech GPS) and Prof. Jonathan
Mitchell (UCLA EPSS).  September 2018 --- August 2019.\\
\item NSF Atmospheric and Geospace Sciences Postdoctoral Research Fellow (NSF award
\#1624740), UCLA Dept. of Atmospheric and Oceanic Sciences (AOS) and Caltech
GPS.  Advisors Prof. Simona Bordoni and Prof. Jonathan Mitchell September 2016
--- August 2018.\\
\end{itemize}
\section*{Education}
\label{sec:org6fcbc82}
\subsection*{Ph.D. | Princeton University | Program in Atmospheric and Oceanic Sciences}
\label{sec:org4491f30}
Conferred September 2016 | Advisor Yi Ming | Committee members: Isaac
Held, Leo Donner, Ming Zhao
\subsection*{B.S. | UCLA | Dept. of AOS and Dept. of Applied Mathematics}
\label{sec:org55f389f}
AOS/Applied Mathematics double major | Conferred June 2011 |
Magna Cum Laude | Phi Beta Kappa | UCLA College Honors
\section*{Publications}
\label{sec:org3534771}
\subsection*{Submitted/under review/in revision}
\label{sec:orge35c7f7}
\begin{enumerate}
\item Mitchell, Jonathan L. and \textbf{Spencer A. Hill}.  "Constraints from invariant
subtropical vertical velocities on the scalings of Hadley cell strength and
downdraft width with rotation rate."  In revision, \emph{Journal of the Atmospheric
Sciences}.  \url{https://arxiv.org/abs/1911.05860}.
\end{enumerate}
\subsection*{Peer-reviewed}
\label{sec:org8ae55b7}
\begin{enumerate}
\item (2020) \textbf{Hill, Spencer A.}, Simona Bordoni, and Jonathan L. Mitchell.
"Axisymmetric Hadley Cell theory with a fixed tropopause temperature rather
than height."  \emph{Journal of the Atmospheric Sciences}, \textbf{77}, 1279-1294.  doi:
\href{https://doi.org/10.1175/JAS-D-19-0169.1}{10.1175/JAS-D-19-0169.1}.
\item (2019) \textbf{Hill, Spencer A.} "Theories for past and future monsoon rainfall
changes."  \emph{Current Climate Change Reports}, \textbf{5}, 160-171.  doi:
\href{https://doi.org/10.1007\%2Fs40641-019-00137-8}{10.1007\%2Fs40641-019-00137-8}.  Online access: \url{https://rdcu.be/bHFCZ}.
\item (2019) \textbf{Hill, Spencer A.}, Simona Bordoni, and Jonathan L. Mitchell.
"Axisymmetric constraints on cross-equatorial Hadley cell extent."
\emph{Journal of the Atmospheric Sciences}, \textbf{76}, 1547-1564.  doi: \href{https://doi.org/10.1175/JAS-D-18-0306.1}{10.1175/JAS-D-18-0306.1}.
\item (2018) \textbf{Hill, Spencer A.}, Juan M. Lora, Norris Khoo, Sean P. Faulk, and
Jonathan M.  Aurnou.  "Affordable rotating fluid demonstrations for
geoscience education: The \emph{DIYnamics} project."  \emph{Bulletin of the
American Meteorological Society}, \textbf{99}, 2529-2538.  doi:
\href{https://doi.org/10.1175/BAMS-D-17-0215.1}{10.1175/BAMS-D-17-0215.1}.
\item (2018) \textbf{Hill, Spencer A.}, Yi Ming, and Ming Zhao.  "Robust responses of the
Sahelian hydrological cycle to global warming."  \emph{Journal of
Climate}, \textbf{31}, 9793-9814.  doi: \href{https://doi.org/10.1175/JCLI-D-18-0238.1}{10.1175/JCLI-D-18-0238.1}.
\item (2018) Smyth, Jane, \textbf{Spencer A. Hill}, and Yi Ming.  "Simulated responses of
the West African monsoon and zonal-mean tropical precipitation to early
Holocene orbital forcing."  \emph{Geophysical Research Letters}, \textbf{45},
12,049-12,057.  doi: \href{https://doi.org/10.1029/2018GL080494}{10.1029/2018GL080494}.
\item (2017) \textbf{Hill, Spencer A.}, Yi Ming, Isaac M. Held, and Ming Zhao.  "A moist
static energy budget-based analysis of the Sahel rainfall response to uniform
oceanic warming."  \emph{Journal of Climate}, \textbf{30}, 5637-5660.  doi:
\href{https://doi.org/10.1175/JCLI-D-16-0785.1}{10.1175/JCLI-D-16-0785.1}.
\item (2017) Brown, Patrick T., Yi Ming, Wenhong Li, and \textbf{Spencer A. Hill}.  "Change
in the magnitude and mechanisms of unforced low-frequency surface temperature
variability in a warmer climate."  \emph{Nature Climate Change}, \textbf{7}, 743-748.
doi: \href{https://doi.org/10.1038/nclimate3381}{10.1038/nclimate3381}.
\item (2017) Jeevanjee, Nadir, Pedram Hassanzadeh, \textbf{Spencer A. Hill}, and Aditi
Sheshadri.  "A perspective on climate model hierarchies."  \emph{Journal
of Advances in Modeling Earth Systems}, \textbf{9}, 1760-1771.  doi: \href{https://doi.org/10.1002/2017MS001038}{10.1002/2017MS001038}.
\item (2015) \textbf{Hill, Spencer A.}, Yi Ming, and Isaac M. Held.  "Mechanisms of forced
tropical meridional energy flux change."  \emph{Journal of Climate}, \textbf{28},
1725-1742.  doi: \href{http://dx.doi.org/10.1175/JCLI-D-14-00165.1}{10.1175/JCLI-D-14-00165.1}.
\begin{itemize}
\item Corrigendum: \url{https://dx.doi.org/10.1175/JCLI-D-16-0485.1}.
\end{itemize}
\item (2012) \textbf{Hill, Spencer A.} and Yi Ming.  "Nonlinear climate response to regional
brightening of tropical marine stratocumulus."  \emph{Geophysical Research Letters},
\textbf{39}, L15707, 5 pp. doi:
\href{http://dx.doi.org/10.1029/2012GL052064}{10.1029/2012GL052064}.
\end{enumerate}
\subsection*{PhD thesis}
\label{sec:org3e18bba}
(2016) \textbf{Hill, Spencer A.} "Energetic and hydrological responses of Hadley
circulations and the African Sahel to sea surface temperature perturbations."
PhD Thesis, Princeton University Program in Atmospheric and Oceanic Sciences.
\subsection*{Non peer-reviewed}
\label{sec:orgfb1406f}
\begin{enumerate}
\item Numerous blog posts for the DIYnamics blog.  Available at
\url{https://diynamics.github.io/blog/author/spencer-hill.html}.
\item (2017) \textbf{Hill, Spencer A.} and Spencer K. Clark.  "What’s needed for the Future
of AOS Python?  Tools for Automating AOS Data Analysis and Management."
Invited guest blog post on "PyAOS" blog.  URL:
\url{http://pyaos.johnny-lin.com/?p=1546}.
\item (2012) \textbf{Hill, Spencer A.}  "A head in the clouds elucidates climate" (book
review of \href{http://press.princeton.edu/titles/9773.html}{\emph{Atmosphere, Clouds, and Climate}} by David Randall). \emph{Science}, \textbf{337},
1 pp., doi: \href{http://dx.doi.org/10.1126/science.1225615}{10.1126/science.1225615}.
\end{enumerate}
\subsection*{Software}
\label{sec:org472a747}
\begin{enumerate}
\item (2018) \textbf{Hill, Spencer A.} and Spencer Clark.  "\texttt{aospy}: automated climate
data analysis and management."  Version 0.3.1.  \url{https://aospy.readthedocs.io}.
doi: \href{https://doi.org/10.5281/zenodo.1490928}{10.5281/zenodo.1490928}.
\item (2016) Hoyer, Stephan et al.  "xarray: v0.8.0."  doi: \href{https://doi.org/10.5281/zenodo.59499}{10.5281/zenodo.59499}.
\end{enumerate}
\section*{Research and Professional Experiences}
\label{sec:org5305561}
\begin{center}
\begin{tabularx}{\textwidth}{rX}
2018 Dec & Co-chair, "Monsoons: Observations, Subseasonal, Seasonal, and Interannual to Decadal Variability, Forecast, Climate Change, and Extremes" session, AGU Fall Meeting, Washington, D.C.\\
2018 Aug & Participant, 17th Swiss Climate Summer School, "Earth system variability through time", Grindelwald, Switzerland\\
2017 June & Chair, \href{https://ams.confex.com/ams/21Fluid19Middle/webprogram/Session43327.html}{"Idealized approaches to the atmospheric and oceanic circulation, Part II"} session, AMS 21st Conference on Atmospheric and Oceanic Fluid Dynamics, Portland, OR\\
2016 Dec & Co-chair, "Tropical circulations and their sensitivities to changes in climate" session, AGU Fall Meeting 2016, San Francisco, CA\\
2016 Nov & Co-chair, "Tropical convection and radiative convective equilibrium" session, WCRP Model Hierarchies Workshop, Princeton, NJ\\
2015 June-Aug & Organizer, Princeton AOS convection journal club\\
2013-2015 & Organizer, GFDL Climate Sensitivity Journal Club\\
2012–2013 & Organizer, Princeton AOS Student/Postdoc Seminar Series\\
2012–2013 & Princeton AOS Program Student Representative to the Faculty\\
2012–2013 & Member, Princeton Energy and Climate Scholars\\
2012 July & Participant, GFDL Summer School on Atmospheric Modeling\\
2011 June-Aug & Research Intern, UCLA California Research Training Program in Computational and Applied Mathematics, Slurry Flows Group\\
2011 Jan & Invited Student Secretary, International Geosphere-Biosphere Program Workshop on Ecosystems Impacts of Geoengineering, Scripps Institution of Oceanography, UCSD, La Jolla, CA\\
2010 June-Aug & Research Intern, NOAA Geophysical Fluid Dynamics Laboratory, Princeton, NJ. Advisor Dr. Yi Ming\\
\end{tabularx}
\end{center}
\section*{Major Honors and Awards}
\label{sec:org604c0aa}
\begin{center}
\begin{tabularx}{\textwidth}{lX}
2019 & Columbia University Earth Institute Postdoctoral Research Fellowship\\
2016 & NSF Atmospheric and Geospace Sciences Postdoctoral Research Fellowship.  NSF Award \#1624740\\
2016 & California Institute of Technology Foster and Coco Stanback Postdoctoral Fellowship, deferred to 2018\\
2013 & U.S. Dept. of Defense National Defense Science and Engineering Graduate Fellowship\\
2012 & Princeton University Elliotte Robinson Little '25 Fellowship\\
2012 & American Meteorological Society Climate Change Travel Scholarship, 92nd AMS Annual Meeting\\
2012 & NSF Graduate Research Fellowship Honorable Mention\\
2009 & National Oceanic and Atmospheric Administration Ernest F. Hollings Undergraduate Scholarship\\
2007 & United States Presidential Scholar.  Honored by President George W. Bush at the White House as part of the Presidential Scholars National Recognition Week.\\
\end{tabularx}
\end{center}
\section*{Teaching \& Mentoring}
\label{sec:orgc62de74}
\subsection*{Princeton Teaching Transcript certification}
\label{sec:org73e6904}
Administered by the Princeton University McGraw Center for Teaching \& Learning.
Requirements include two-day teacher training, lectures and workshops on
pedagogy, and video recording and subsequent analysis of teaching as a TA.
Completed August 2016.
\subsection*{Teaching Assistant}
\label{sec:org2d0c988}
Princeton University, Fall 2014, Geosciences 361, "Physics of Earth: The
Habitable Planet."  Professor George Philander.
\subsection*{Mentorship}
\label{sec:org92fdb5c}
\begin{itemize}
\item Advisor to Barnard University undergraduate student Valentina Rojas-Posada on
project linking Indian summer rainfall variability with societal outcomes
(Fall 2020)
\item Advisor to UCLA undergraduate student Micah Kim for independent research
course work on the "aospy" software package (Fall 2017)
\item Assistant mentor to UCLA undergraduate students for work on the DIYnamics
portable rotating tank science outreach project: Norris Khoo (2017), Juliet
Olsen (Fall 2016)
\item Assistant mentor to summer interns at NOAA GFDL: Jane Smyth (2015), Marjahn
Finlayson (2014), Colin Raymond (2013)
\end{itemize}
\subsection*{Guest lectures}
\label{sec:org07ac869}
\begin{center}
\begin{tabularx}{\textwidth}{lX}
\textit{[2018-11-29 Thu]} & "Thermal wind balance: lab demonstrations"  UCLA AOS 200A (graduate level): Introduction to Atmospheric and Oceanic Fluid\\
\textit{[2018-11-15 Thu]} & "The general circulation of the atmosphere: momentum"  UCLA AOS 200A (graduate level): Introduction to Atmospheric and Oceanic Fluid\\
\textit{[2018-11-13 Tue]} & "The general circulation of the atmosphere: energetics"  UCLA AOS 200A (graduate level): Introduction to Atmospheric and Oceanic Fluid\\
\textit{[2014-12-11 Thu]} & "Climate modeling and climate change projections"  Princeton University GEO 361 (upper-division undergraduate):  Physics of Earth: The Habitable Planet.\\
\textit{[2014-10-14 Tue]} & "Weather: From orbital order, atmospheric chaos"  Princeton University GEO 361 (upper-division undergraduate):  Physics of Earth: The Habitable Planet.\\
\end{tabularx}
\end{center}
\section*{Public Outreach}
\label{sec:orgb748a6e}
\subsection*{"DIYnamics" project}
\label{sec:org0043227}
\subsubsection*{Overview}
\label{sec:org5643b19}
Developing inexpensive, easy-to-assemble rotating tank platforms for use in
teaching fundamental principles of Earth science at levels from K-12 science
classes to AOS graduate courses.  Materials include the kits themselves, PDF
assembly instruction, instructional videos, and a website:
\url{https://diynamics.github.io/}

\subsubsection*{Press}
\label{sec:org43b2563}
\begin{itemize}
\item UCLA press release (\href{http://newsroom.ucla.edu/releases/a-50-do-it-yourself-device-designed-at-ucla-makes-science-fun-for-students-of-all-ages}{\texttt{link}})
\item Story on UCLA Physical Sciences Division website (\href{https://www.physicalsciences.ucla.edu/spinlab/}{\texttt{link}})
\item Listed on official UCLA Physical Sciences Division Outreach page (\href{https://www.physicalsciences.ucla.edu/outreach/}{\texttt{link}})
\end{itemize}

\subsubsection*{Outreach events}
\label{sec:org06f60ed}
Organized and/or participated in numerous DIYnamics public outreach events from
2017 onward, all documented on the DIYnamics blog: \url{https://diynamics.github.io/pages/blog.html}
\subsection*{Outreach activities prior to DIYnamics}
\label{sec:org378e616}
\begin{center}
\begin{tabularx}{\textwidth}{lX}
\textit{[2015-06-19 Fri]} & "Introduction to climate models."  20 minute presentation to New Jersey Japanese School during their visit to NOAA Geophysical Fluid Dynamics Laboratory, Princeton, NJ.\\
\textit{[2015-04-10 Fri]} & "Introduction to weather and climate."  45 minute presentation + Q\&A to 7th grade class at Forrestdale Middle School, Rumson, NJ.  Co-presented with Sarah Schlunegger.\\
\end{tabularx}
\end{center}
\section*{Software: \texttt{aospy}}
\label{sec:orgcea0d3e}
\subsection*{Overview}
\label{sec:org7424b7f}
Created and actively develop "aospy," an open source Python package for
automating computations that use gridded climate or weather data.  Enables
executing multiple calculations in parallel using the permutation of an
arbitrary number of simulations, variables to be computed, date ranges, and many
other parameters.  Results get saved in a highly organized directory tree as
netCDF file.  Actively used by researchers at multiple institutions.

\subsection*{Resources}
\label{sec:orgef4abce}
\begin{itemize}
\item Documentation: \url{https://aospy.readthedocs.io}
\item Source code: \url{https://github.com/spencerahill/aospy}
\end{itemize}
\section*{Reviewing}
\label{sec:org14b66ad}
\begin{itemize}
\item Proposal reviewer for NSF Climate and Large-scale Dynamics, NSF EarthCube,
NASA Juno Participating Scientist Program
\item Article reviewer for \emph{Nature Climate Change}, \emph{npj Climate and Atmospheric
Science}, \emph{Journal of Advances in Modeling Earth Systems}, \emph{Journal of
Climate}, \emph{Quarterly Journal of the Royal Meteorological Society},
\emph{Geophysical Research Letters}, \emph{Journal of the Atmospheric Sciences},
\emph{Geoscientific Model Development}, \emph{Climatic Change}, \emph{Climate Dynamics},
\emph{Journal of Geophysical Research - Atmospheres}, GFDL internal manuscript
review.
\end{itemize}

\section*{Presentations}
\label{sec:org7b25e71}
\subsection*{Invited}
\label{sec:org1bd8819}
\begin{center}
\begin{tabularx}{\textwidth}{lX}
\textit{[2019-12-10 Tue]} & "Toward an analytical, predictive theory for the location of Hadley and monsoonal cell ascending branches."  Invited conference talk.  American Geophysical Union Fall Meeting.  San Francisco, CA.\\
\textit{[2019-01-07 Mon]} & "Robust responses of the Sahelian hydrological cycle to global warming."  Invited seminar.  UCSB Earth Research Institute Monthly Climate Meeting.  Santa Barbara, CA.\\
\textit{[2018-11-07 Wed]} & "Back to the basics of monsoons, Hadley cells, and rotating tanks."  Invited seminar.  UCLA AOS formal seminar series.\\
\textit{[2016-11-12 Sat]} & "\texttt{infinite-diff} and \texttt{animal-spharm}: \texttt{xarray}-based finite differencing and spherical harmonics."  Invited conference talk.  Columbia University AOS-Python workshop.  New York, NY.\\
\textit{[2016-10-13 Thu]} & "The fate of rainfall in the African Sahel under global warming."  Invited seminar.  Westmont College, Santa Barbara, CA.\\
\textit{[2015-05-21 Thu]} & "Towards constraining Sahel rainfall responses to global mean temperature changes."  Invited conference talk.  Linde Center for Global Environmental Science "Monsoons: Past, Present and Future" workshop, California Institute of Technology, Pasadena, CA.\\
\textit{[2015-03-06 Fri]} & "Mechanisms of forced ITCZ shifts and of rainfall responses in the African Sahel to SST warming."  Invited seminar.  New York University AOS student seminar series.\\
\textit{[2014-10-09 Thu]} & "Mechanisms of forced ITCZ shifts and Sahelian drought in GCMs."  Invited seminar.  Yale University, New Haven, CT.\\
\end{tabularx}
\end{center}

\subsection*{Others}
\label{sec:org111f822}
\begin{center}
\begin{tabularx}{\textwidth}{lX}
(planned) 2021-01 & "Connecting sub-India, sub-seasonal monsoon rainfall variability with all-India, all-summer monsoon rainfall." AMS Annual Meeting.  (virtual)\\
(planned) 2020-12 & "Sub-India summer monsoon rainfall variability and its implications for all-India summer monsoon rainfall prediction."  AGU Fall Meeting.  (virtual)\\
\textit{[2019-12-10 Tue]} & "Simulated polar amplification and its causes on decadal to millennial timescales."  Poster.  AGU Fall Meeting.  San Francisco, CA.\\
\textit{[2019-09-11 Wed]} & "Towards transient simulation of the Green Sahara onset and demise through idealized modeling of vegetation-land-atmosphere interactions."  Poster.  Lamont-Doherty Earth Observatory Postdoc Symposium.  Palisades, NY.\\
\textit{[2019-06-25 Tue]} & "Modernizing Axisymmetric Hadley Cell and Monsoon Theory"  Oral.  AMS 22nd Conference on Atmospheric and Oceanic Fluid Dynamics.  Portland, ME.\\
\textit{[2018-12-13 Thu]} & "What sets the locations of the solsticial cross-equatorial Hadley cell edges?"  Oral.  AGU Fall Meeting.  Washington, DC.\\
\textit{[2018-08-28 Tue]} & "Towards transient simulation of the Green Sahara onset and demise through idealized modeling of vegetation-land-atmosphere interactions."  Poster.  17th Swiss Climate Summer School, "Earth system variability through time", Grindelwald, Switzerland.\\
\textit{[2018-04-16 Mon]} & "What Determines the ITCZ Position During Solsticial Seasons on Earth and Other Planets?"  Oral.  AMS 33rd Conference on Hurricanes and Tropical Meteorology.  Ponte Vedra, FL.\\
\textit{[2018-02-16 Fri]} & "Robust responses of the Sahelian hydrological cycle to global warming."  Oral.  Columbia University/Lamont-Doherty Earth Observatory formal seminar series.  Palisades, NY.\\
\textit{[2018-02-14 Wed]} & "What limits the ITCZ's poleward extent during summer?"  Oral.  NOAA Geophysical Fluid Dynamics Laboratory informal seminar series.  Princeton, NJ.\\
\textit{[2017-12-14 Thu]} & "Dry Rainbelts: Understanding Boundary Layer Controls on the ITCZ Using a Dry Dynamical Core."  Oral.  AGU Fall Meeting.  New Orleans, LA.\\
\end{tabularx}
\end{center}

\begin{center}
\begin{tabularx}{\textwidth}{lX}
\textit{[2017-07-19 Wed]} & "Towards transient simulation of the Green Sahara onset and demise through idealized modeling of vegetation-land-atmosphere interactions."  Poster.  Gordon Research Conference on Radiation and Climate.  Bates College, Lewiston, ME.\\
\textit{[2017-06-27 Tue]} & "Control of convergence zone migrations by planetary parameters."  Poster.  AMS 21st Conference on Atmospheric and Oceanic Fluid Dynamics.  Portland, OR.\\
\textit{[2017-01-24 Tue]} & "Automate your climate and weather data analysis with aospy."  Oral.  AMS Annual Meeting, Seattle, WA.\\
\textit{[2017-01-24 Tue]} & "Energetic and precipitation responses in the Sahel to sea surface temperature perturbations."  Oral.  AMS Annual Meeting, Seattle, WA.\\
\textit{[2016-12-16 Fri]} & "Robust drying influence of mean ocean surface warming on The Sahel and implications for constraining future rainfall change."  Poster.  AGU 2016 Fall Meeting, San Francisco, CA.\\
\textit{[2016-11-09 Wed]} & "Energetic and precipitation responses in the Sahel to sea surface temperature perturbations."  Oral.  New York University Center for Atmosphere Ocean Science formal seminar series.  New York, NY.\\
\textit{[2016-11-02 Wed]} & "A hierarchy of perturbation complexites: Case study of Sahel rainfall response to global warming."  Poster.  WCRP Model Hierarchies Workshop, Princeton University, Princeton, NJ.\\
\textit{[2016-10-26 Wed]} & "Tropical energetic and precipitation responses to sea surface temperature perturbations: Zonal mean and the African Sahel."  Oral.  Caltech GPS formal seminar series.\\
\textit{[2016-10-05 Wed]} & "Tropical energetic and precipitation responses to sea surface temperature perturbations: Zonal mean and the African Sahel."  Oral.  UCLA AOS formal seminar series.\\
\textit{[2015-12-14 Mon]} & "Towards constraining future rainfall in the Sahel using the moist static energy budget." Oral.  AGU 2015 Fall Meeting, San Francisco, CA.\\
\textit{[2015-03-13 Fri]} & "Radiative and dynamical controls on the Sahel rainfall response to uniform ocean warming."  Oral.  Princeton AOS dynamics seminar series.\\
\textit{[2015-01-06 Tue]} & "Convection scheme, cloud, and stability effects on Sahel rainfall response to uniform warming."  Poster.  AMS Annual Meeting, Phoenix, AZ.\\
\textit{[2014-12-15 Mon]} & "Convection scheme, cloud, and stability effects on Sahel rainfall response to uniform warming."  Poster.  AGU Fall Meeting, San Francisco, CA.\\
\textit{[2014-06-19 Thu]} & "Mechanisms of forced tropical meridional energy flux change."  Poster presentation.  Latsis Symposium, ETH Zurich, Zurich, Switzerland.\\
\textit{[2014-02-05 Wed]} & "Mean and extreme tropical precipitation changes caused by the uniform and spatially varying components of anthropogenic forcing."  Oral presentation.  AMS 2014 Annual Meeting, Atlanta, GA.\\
\textit{[2013-12-13 Fri]} & "Mechanisms of forced tropical meridional energy flux change."  Oral presentation.  AGU 2013 Fall Meeting, San Francisco, CA.\\
\textit{[2013-11-02 Sat]} & "Mechanisms of forced tropical meridional energy flux change."  Oral presentation.  Graduate Climate Conference, Woods Hole Oceanographic Institution, Woods Hole, MA.\\
\textit{[2013-07-09 Tue]} & "Mechanisms of forced tropical meridional energy flux change."  Poster presentation.  Gordon Research Conference, Colby-Sawyer College, New London, NH.\\
\textit{[2012-02-28 Tue]} & "Climate response to a geoengineered brightening of subtropical marine boundary clouds."  Oral.  Princeton AOS Program Student/Postdoc Seminar Series.\\
\textit{[2012-01-22 Sun]} & "Climate response to a geoengineered brightening of subtropical marine boundary clouds."  Poster.  11th Annual Student Conference at the AMS Annual Meeting, New Orleans, LA.\\
\textit{[2011-11-11 Fri]} & "Climate response to a geoengineered brightening of subtropical marine boundary clouds."  Oral.  Princeton University Department of Geosciences Graduate Research Symposium, Princeton, NJ.\\
\textit{[2010-12-16 Thu]} & "Climate response to a geoengineered brightening of subtropical marine boundary clouds."  Poster.  Session GC31A: "Can We Counteract Global Warming?" American Geophysical Union Fall Meeting.  San Francisco, CA.\\
\textit{[2010-10-26 Tue]} & "Climate response to a geoengineered brightening of subtropical marine boundary clouds."  Oral.  Special Symposium on Aerosols in Geoengineering at the American Association for Aerosol Research 29th Annual Conference.  Portland, OR.\\
\textit{[2010-08-03 Tue]} & "Investigating climate response to geoengineering using a global climate model."  Oral.  National Oceanic and Atmospheric Administration Office of Education Science Symposium, Silver Spring, MD.\\
\end{tabularx}
\end{center}
\end{document}